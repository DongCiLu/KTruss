\section{Related Works}
\label{relatedwork} 

Our work falls in the category of cohesive subgraph mining \cite{sozio2010community,cui2014local,bera2018maximal} such as community detection and community search. It is closely related to an inspiring work \cite{huang2014querying}, which introduce the model of k-truss community based on triangle connectivity. 
The k-truss concept is first introduced by the work \cite{cohen2008trusses}. However, the original definition of a k-truss lacks the connectivity constraint so that a k-truss may be a unconnected subgraph. 
%\cite{huang2014querying} introduce the model of k-truss community based on triangle connectivity. 
The notion of triangle connected k-truss communities is also referred to as $k-(2,3)$ nucleus in \cite{sariyuce2017nucleus} where they propose an approach based on the disjoint-set forest to speed up the process of nucleus decomposition. 

Due to expensive triangle enumeration, triangle-connected k-truss communities have slow computation efficiency, especially for vertices belonging to large k-truss communities. To speed up the community search based on this model, an index structure called the TCP index is proposed in \cite{huang2014querying}, where each vertex holds its maximum spanning forest based on the edge trussness of their ego-network. 
%Triangle connected k-truss communities mitigate the "free-rider" issue but at the cost of slow computation efficiency especially for vertices belongs to large k-truss communities so that they are not able to meet requirements of queries in some cases. 

The Equitruss index \cite{akbas2017truss} uses a super-graph based on truss-equivalence as an index to speed up the single-vertex k-truss community search.  
%stores a partition of the original graph based on the notion of k-truss equivalence. 
The Equitruss index is similar to our index structure in the sense that it also uses a super-graph for the index. However, the vertex in the super-graph of the Equitruss index is a subgraph of a k-truss community while an edge represents the triangle connectivity. Our \twolevelindex{} contains a more compact super-graph in which a vertex contains a k-truss community and edges represent the k-truss community containment relations.    The difference in the size of the super-graph leads to different performance for various local k-truss community queries, especially for the \toplevelprob{} queries. 

 
%Many previous works have studied this problem based on various cohesive subgraph models, such as clique (\cite{bron1973algorithm,rossi2014fast}), k-core (\cite{shin2016corescope,barbieri2015efficient,li2017discovering}), k-truss (\cite{huang2014querying,wang2012truss,cohen2008trusses,huang2015approximate,huang2016truss,zheng2017finding}), k-plex (\cite{wangquery}) and quasi-clique (\cite{tsourakakis2013denser, lee2016query}). The k-truss concept is first introduced by the work \cite{cohen2008trusses}. However, the original definition of a k-truss lacks the connectivity constraint so that a k-truss may be a unconnected subgraph. \cite{huang2014querying} introduce the model of k-truss community based on triangle connectivity. The notion of triangle connected k-truss communities is also referred to as $k-(2,3)$ nucleus in \cite{sariyuce2016fast, sariyuce2017nucleus} where they propose an approach based on the disjoint-set forest to speed up the process of nucleus decomposition. K-truss decomposition is also studied in \cite{huang2016truss, zou2017truss} for probabilistic graphs. 

%An $\alpha$-adjacency $\gamma$-quasi-$k$-clique model is introduced by \cite{cui2014local} for online searching of overlapping communities. $\rho$-dense core is a pseudo clique recently introduced by \cite{koujaku2016dense} that can deliver the optimal solution for graph partition problems. The pattern of k-core structures is studied by \cite{shin2016corescope} for applications such as finding anomalies in real world graphs, approximate degeneracy of large-scale graphs and so on. \cite{li2015influential} introduce a novel community model called k-influential community based on the concept of k-core, which can capture the influence of a community. %\cite{chen2016efficient,chang2017mathsf,li2017finding,bi2017optimal,li2017most} also study this.
%\cite{wu2015robust} systematically study the existing goodness metrics and provide theoretical explanations on why they may cause the free rider effect. %We further develop a query biased node weighting scheme to reduce the free rider effect.
%\cite{huang2015approximate} try to address the "free rider" issue by finding communities that meet cohesive criteria with the minimum diameter.
%There are also many works studied community related problems on attribute graph (\cite{fang2016effective,shang2016agar,shang2017attribute,huang2017attribute}), \eg finding communities that satisfy both structure cohesiveness, \ie its vertices are tightly connected, and keyword cohesiveness, \ie its vertices share common keywords.
%\cite{interdonato2017local} design a framework for local community detection in multilayer networks.