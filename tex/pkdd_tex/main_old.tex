%%%%%%%%%%%%%%%%%%%%%%% file template.tex %%%%%%%%%%%%%%%%%%%%%%%%%
%
% This is a general template file for the LaTeX package SVJour3
% for Springer journals.          Springer Heidelberg 2010/09/16
%
% Copy it to a new file with a new name and use it as the basis
% for your article. Delete % signs as needed.
%
% This template includes a few options for different layouts and
% content for various journals. Please consult a previous issue of
% your journal as needed.
%
%%%%%%%%%%%%%%%%%%%%%%%%%%%%%%%%%%%%%%%%%%%%%%%%%%%%%%%%%%%%%%%%%%%
%
\RequirePackage{fix-cm}
%
%\documentclass{svjour3}                     % onecolumn (standard format)
%\documentclass[smallcondensed]{svjour3}     % onecolumn (ditto)
\documentclass[smallextended,natbib]{svjour3}       % onecolumn (second format)
%\documentclass[twocolumn]{svjour3}          % twocolumn
%
\smartqed  % flush right qed marks, e.g. at end of proof
%
\usepackage{graphicx}
%
% \usepackage{mathptmx}      % use Times fonts if available on your TeX system
%
% insert here the call for the packages your document requires
\usepackage{booktabs} % For formal tables
\usepackage{amsmath}
\usepackage[noend]{algpseudocode}
\usepackage[flushleft]{threeparttable}
\usepackage{subfigure}
\usepackage{comment}
\usepackage{color}
\usepackage{soul}
\usepackage{hyperref}
\usepackage{url}
\usepackage{multirow}
\usepackage[lined,linesnumbered,ruled]{algorithm2e}
\usepackage[flushleft]{threeparttable}
\usepackage{mfirstuc}
%\usepackage{latexsym}
% etc.
%
% please place your own definitions here and don't use \def but
% \newcommand{}{}
\makeatletter
\def\th@plain{%
  \thm@notefont{}% same as heading font
  \itshape % body font
}
\def\th@definition{%
  \thm@notefont{}% same as heading font
	\itshape
}
\makeatother

\newtheorem{Def}{Definition}%[section]
\newcommand{\Defautorefname}{Definition}
\newtheorem{Thm}{Theorem}%[section]
\newcommand{\Thmautorefname}{Theorem}
\newcommand{\ie}{i.e., }
\newcommand{\eg}{e.g., }
\newcommand{\usernote}[1]{\textbf{\color{blue}[***** #1 *****]\\}}

% Problem specific macros
\newcommand{\probdef}{k-truss community search}
\newcommand{\Probdef}{K-truss community search}
\newcommand{\ProbDef}{\expandafter\capitalisewords\expandafter{\probdef}}
\newcommand{\inducedgraph}{induced MST graph}
\newcommand{\Inducedgraph}{Induced MST graph}
\newcommand{\InducedGraph}{\expandafter\capitalisewords\expandafter{\inducedgraph}}
\newcommand{\treeindex}{tree-structured community graph}
\newcommand{\Treeindex}{Tree-structured community graph}
\newcommand{\TreeIndex}{\expandafter\capitalisewords\expandafter{\treeindex}}

%
% Insert the name of "your journal" with
\journalname{Data Mining and Knowledge Discovery}
%
\begin{document}

\title{Fast Truss Community Query in Large-scale Dynamic Graphs%\thanks{Grants or other notes
%about the article that should go on the front page should be
%placed here. General acknowledgments should be placed at the end of the article.}
}
%\subtitle{Do you have a subtitle?\\ If so, write it here}

%\titlerunning{Short form of title}        % if too long for running head

\author{Zheng Lu         \and
				Yunhe Feng			 \and
        Qing Cao %etc.
}

%\authorrunning{Short form of author list} % if too long for running head

\institute{Zheng Lu \at
							Electrical Engineering \& Computer Science, University of Tennessee \\
              % Tel.: +123-45-678910\\
              % Fax: +123-45-678910\\
              \email{zlu12@vols.utk.edu}           %  \\
%             \emph{Present address:} of F. Author  %  if needed
           \and
					 Yunhe Feng \at
							Electrical Engineering \& Computer Science, University of Tennessee \\
							\email{yfeng14@vols.utk.edu} 
           Qing Cao \at
							Electrical Engineering \& Computer Science, University of Tennessee \\
							\email{cao@utk.edu} 
}

\date{Received: date / Accepted: date}
% The correct dates will be entered by the editor


\maketitle

\begin{abstract}
Recently, there has been significant interest in the study of the community search problem in social and information networks: given one or more query nodes, find densely connected communities containing the query nodes. However, most existing algorithms require linear computational time to the size of the found community for each specific K value. Therefore, state-of-the-art algorithms have limited scalability in large scale graphs, where communities grow to millions of edges.

In this paper, given an undirected graph G and a set of query nodes $Q$, we formulate a new problem call truss community identity search, which can help answer various query types with multiple query vertices such as max-k truss query, any-k truss query as well as k-truss query. This new proposed problem can be used in many applications as building blocks for community related queries. To solve this problem and k-truss community search efficiently, we propose a 2-level index structure for fast query process. Not only our index structure can support various query types with multiple query vertices. We ourperform state-of-the-art truss index for single vertex k-truss community search. We also develop an efficient algorithm to construct such an index. Extensive experiments on real-world networks show the effectiveness and efficiency of our algorithms. 
\keywords{K-truss \and dynamic graph \and query process}
% \PACS{PACS code1 \and PACS code2 \and more}
% \subclass{MSC code1 \and MSC code2 \and more}
\end{abstract}

\section{Introduction}
\label{introduction}

Graphs are naturally used to model many real-world networks, \eg online social networks, biological networks, collaboration and communication networks. As community structures are commonly found in real-world networks, community related problems have been widely studied in the literature, such as community detection (\cite{newman2004finding, xie2013overlapping}) and community search (\cite{huang2014querying, akbas2017truss, huang2015approximate, lee2016query, sozio2010community, cui2014local, li2015influential, barbieri2015efficient}), and have found a wide range of applications (\cite{durmaz2017frequent,zong2015behavior,yin2017taming}). Triangles are known as fundamental building blocks of networks. K-truss as a definition of cohesive subgraph based on triangles of a graph, requires that each edge be contained in at least $(k - 2)$ triangles within this subgraph. The low computation cost of k-truss makes it suitable to scale to large-scale graphs.
The original definition of a k-truss lacks the connectivity constraint so that a k-truss may be an unconnected subgraph. \cite{huang2014querying} introduce the model of k-truss community based on triangle connectivity.

Community search, a query-dependent variant of community detection, attracts more attention as it enables targeted community discovery around given seed vertices of interest and is faster to process in the sense that the runtime does not depend on the size of the graph. However, all the relevant communities still have to be exhaustively identified, leading to excessive computation time/space if details of communities are not of interest. There are various types of queries that are useful in real-world applications involving communities that do not require details of communities. 
%The local community queries, such as community search, 
We can classify local community queries, such as community search, into two categories according to the level of information required to answer a query, the \toplevelprob{} query and the \bottomlevelprob{} query. The \toplevelprob{} query requires only relation information between different communities of interest. For example, "Do query vertices belongs to the same communities?", "What is the level of cohesiveness among all query vertices?". It is possible to process this type of queries by only examining relations between relavant communities to which query vertices belong without diving into inside structures of them. Another type of queries, the \bottomlevelprob{} query, requires edge level information to process, for example, the widely studied community search problem or finding boundaries of a target community. One has to know edge level structures inside relevant communities to be able to answer such queries.

\begin{figure}[ht]
    \centering
    \includegraphics[width=0.8\linewidth]{./figures/illustration_main.png}
    \caption{Two layer index structure for k-truss community queries.}
    \label{fig:illustration_main}
		\vspace{-0.2 in}
\end{figure}

Local community queries, such as community search, based on the k-truss community model (\cite{huang2014querying}) can benefit from compact index structures constructed from pre-computed results due to the low computational cost of the k-truss community model. Previous works mainly focused on community search problem of a single query vertex (\cite{huang2014querying, akbas2017truss}). In this paper, we propose a novel 2-level index structure, to support both the \toplevelprob{} and the \bottomlevelprob{} k-truss community query. An overview of our \twolevelindex{} is shown in \autoref{fig:illustration_main}. The top level index is a super-graph with vertices represent unique k-truss communities and edges represent containment relations between k-truss communities. For the bottom level index, We introduce a new type of graph called triangle derived graph that translates triangle connectivity in a graph to edge connectivity for fast k-truss community traversal. We store a maximum spanning forest of the triangle derived graph generated from the underlying graph that preserves the detailed edge level structure of k-truss communities in the bottom level index. 
%We combine the top level index and bottom level index together by partition the bottom level index based on 
%We partitioned this maximum spanning tree according to k-truss communities and the subgraph belonging to each k-truss community separatel. 
The super-graph forms a forest, and we can use simple $union$ and $intersection$ operations to locate relevant k-truss communities of a given query. 
To handle local k-truss community queries, we can use simple $union$ and $intersection$ operations on the top level index to efficiently locate target communities of a query. These communities can be used to answer \toplevelprob{} queries directly or handed to the bottom level index to processing inner-community details for \bottomlevelprob{} queries. The bottom level index is only used for \bottomlevelprob{} queries that using edge level information to further process relevant communities provided by the top level index. For example, in a community search query, we can first use the top level index to find target k-truss communities that contain all query vertices and then use the bottom level index to retrieve edges contained in each target k-truss community. The \twolevelindex{} proposed in this paper can efficiently process both single-vertex queries and multiple-vertex queries. We proved our index and query process to be theoretically optimal and showed its efficiency in practice for both \toplevelprob{} and \bottomlevelprob{} k-truss community queries on real-world graphs and demonstrate its performance by comparing with state-of-the-art methods, the TCP index (\cite{huang2014querying}) and the Equitruss index (\cite{akbas2017truss}). For reproducibility, we make the source code available online.\footnote{https://github.com/DongCiLu/KTruss1 34}

Our contribution can be summarized as follows.
\begin{itemize}
	\item We categorize the local k-truss community queries into the \toplevelprob{} query and the \bottomlevelprob{} query based on the information required to answer each type of query. 
	%which is supported by various query types. The k-truss community identity search is efficient for many real-world applications and is much more efficient than k-truss community search based approach.
	\item We develop a 2-level index structure that can efficiently process both the \toplevelprob{} and the \bottomlevelprob{} k-truss community query. The top level index contains a super-graph for locating target communities of a given query. The bottom level index preserves the edge level triangle connectivity for detailed search of inner-community structures.  
	%\item We design an efficient bottom-up index construction algorithm for our 2-level index structure. The time and space complexity is $O(m\log{m})$ and $O(m)$ respectively.
	\item We perform extensive experiments on our 2-level index on large-scale real-world graphs and compare it with state-of-the-art index structures. We can process \toplevelprob{} queries in the range of hundreds of microseconds to less than a second. We can process \bottomlevelprob{} queries in the range of few seconds to hundreds of seconds for highest degree vertices within large communities. %Our index The results show that our index is not only much compact and efficient for k-truss community search queries but can also support various query types.
\end{itemize}

The rest of this paper is organized as follows. Section~\ref{preliminary} provides notations and definitions used in this paper. We design a novel 2-level index structure in Section~\ref{index}. Section~\ref{query} discusses the query process on the proposed index structure. The evaluations of our algorithm are in Section~\ref{evaluation}.  We discuss previous works in Section~\ref{relatedwork} and conclude our work in Section~\ref{conclusion}.


\section{Preliminaries}
\label{preliminary}

In our problem, we consider a graph $G = (V,E)$, with $n = |V|$ vertices and $m = |E|
$ edges. We denote the set of neighbors of a vertex $v$ by $N(v)$, i.e., $N(v)= {u \in V :(v, u) \in E}$, and the degree of $v$ by $d(v)= |N(v)|$. A triangle in G is a cycle of length 3. Let $u, v, w \in V$ be the three vertices on the cycle, and we denote this triangle by $\triangle uvw$.

The support of an edge $e(u, v) \in E$ in G, denoted by $sup(e,G)$, is defined as $|{\triangle uvw : w \in V}|$. When the context is obvious, we replace sup(e,G) by sup(e).

Given two triangles ${\triangle}_{1}$, ${\triangle}_{2}$ in G, they are adjacent if they share a common edge, denoted by ${\triangle}_{1} \cap {\triangle}_{2} \neq \emptyset$. Two triangles are triangle connected if there exist a series of triangles ${\triangle}_{1}$, ..., ${\triangle}_{n}$ in G such that for $1 \le i < n, {\triangle}_{i} \cap {\triangle}_{i+1} \neq \emptyset$.

The trussness of a subgraph $H \in G$ is the minimum support of an edge in H, denoted by $\tau(H) = min{sup(e,H): e \in E(H)}$.  The trussness of an edge $e \in E(G)$ is defined as $\tau(e)=max_{H \in G}{τ(H): e \in E(H)}$. %The trussness of a triangle is defined as $τ(\triangle) = min{τ(e,H): e \in E(\triangle)}$

Given a graph G and an integer $k \ge 2$, G' is a k-truss community, if G' satisfies the following three conditions: 

(1) K-Truss. G' is a subgraph of G, denoted as $G' \subseteq G$, such that $\forall e \in E(G'), sup(e,G') ≥ (k − 2)$;

(2) Edge Connectivity. $\forall e1,e2 \in E(G'), \exists {\triangle}_{1}, {\triangle}_{2}$ in G' such that $e1 \in {\triangle}_{1}, e2 \in {\triangle}_{2}$, then either ${\triangle}_{1} = {\triangle}_{2}$, or ${\triangle}_{1}$ is triangle connected with ${\triangle}_{2}$ in G';

(3) Maximal Subgraph. G' is a maximal subgraph satisfying conditions (1) and (2). That is, $G'' \in G$, such that $G' \in G''$, and G'' satisfies conditions (1) and (2).

Problem Definition. The problem of studied in this paper is defined as follows. Given a graph $G(V,E)$, a set of query vertices $Q \in V$, find all truss communities containing $Q$ with maximum $k$, a specific $k$ or any possible $k$. 
\section{Indexed \ProbDef}
\label{index}

We propose to solve \probdef{} problem using an index based approach. This section describes how to process a \probdef{} query on a static graph, including \inducedgraph{} construction, creating \treeindex{}, performing various kind of queries on the preprocessed index. In the next section, we describe index update procedure on dynamic graphs.

\subsection{\InducedGraph{}}

We first design an \inducedgraph{} then propose the query algorithm based on it.

\vskip 0.1in \noindent \textbf{\InducedGraph{} Construction.} We first compute the edge trussness of graph $G_o$ and then construct a new graph $G_m$, which we called \inducedgraph{}, based on the graph $G_o$ and its edges' trussness. We define the \inducedgraph{} as follows.

\begin{Def}[\inducedgraph{}]
The \inducedgraph{} is a weighted maximum spanning forest that each edge $e$ in $G_o$ is represented as a vertex $x$ in $G_m$. An edge $y$ in $G_m$ represents that the two edges, which are represented by the two adjacent vertices of $y$, are contained in the same triangle in $G_o$. The weight of the each vertex in $G_m$ is its represented edge's trussness in $G_o$. The weight of each edge in $G_m$ is the lowest edge trussness of its related triangle's edges in $G_o$.
\label{def:\inducedgraph{}}
\end{Def}

We denote $G_{m}^{\prime}$ as the graph that is constructed the same way as $G_m$ but with all triangles in $G_o$ as edges, \ie $G_m$ is the maximum spanning forest of $G_{m}^{\prime}$. We refer to lowest edge trussness of a triangle as the weight of the triangle. 

We have the following theorem for vertex weights and edge weights in \inducedgraph{} $G_m$.

\begin{Thm}
In \inducedgraph{} $G_m$, for each vertex $x$ and each of its adjacent edge $y$, we have $w_x \ge w_y$.
\label{thm:\inducedgraph{}_vertex_trussness}
\end{Thm}

\begin{proof}
According to \autoref{def:\inducedgraph{}}, $w_x$ is the trussness of the represented edge $e$ in $G_o$ while $w_y$ is the lowest trussness of edges in the represented triangle $\triangle$ in $G_o$. We have $\tau_{e} \ge \tau_{\triangle}$, therefore, $w_x \ge w_y$.
\end{proof}

%\begin{algorithm}
	%\KwData{$G_{o}(V,E)$}
	%\KwResult{edge trussness $\{\tau_{e}, e \in E\}$}
	%\BlankLine
	%compute $s_e$ for each edge $e$ and sort in ascending order\;
	%$k \gets 2$\;
	%\While{$\exists e \in E$}{
		%\While{$\exists e$ such that $s_e \le k - 2$}{
			%$e \gets$ lowest support edge $(u, v)$\;
			%$u \gets$ lower degree end of $e$\;
			%\For{$w \in N_u$}{
				%\If{$(v,w) \in E$}{
					%$s_{(u,w)} \gets s_{(u,w)} + 1$\;
					%$s_{(v,w)} \gets s_{(v,w)} + 1$\;
					%reorder the sorted list of edges\;
				%}
			%}
			%$\tau_e \gets k$\;
			%remove $e$ from $E$\;
		%}
		%$k \gets k + 1$\;
	%}
	%\Return{$\{\tau_{e}, e \in E\}$}
	%\caption{Truss Decomposition}\label{alg:truss_decomposition}
%\end{algorithm}

%The truss decomposition algorithm \cite{wang2012truss} is used to compute trussness of all edges in $G_o$. For the ease of reader, we show the algorithm in \autoref{alg:truss_decomposition}. The algorithm first computes support of each edge by counting number of triangles it belongs to. Then, for $k$ starting at $2$, it iteratively removes the edge with lowest support from the graph and update supports of other edges accordingly. The removed edge is given the edge trussness of $k$. The algorithm increases $k$ by $1$ if there is no more edge has a support lower than $k - 2$. 
%The algorithm takes $O(\sum{(u,v) \in E}\min{d_u, d_v}$ time and $O(m)$ space for computing the trussness of all edges.

\begin{algorithm}
	\KwData{$G_{o}(V_{o},E_{o})$, edge trussness $\{\tau_{e}, e \in E_{o}\}$}
	\KwResult{$\inducedgraph{} G_{m}(V_{m}, E_{m})$}
	\BlankLine
	$visited \gets \emptyset$\;
	\For{$(u,v) \in E_{o}$}{
		suppose $u$ is the lower degree end of $(u,v)$\;
		$V_{m} \gets V_{m} \bigcup \{(u,v), \tau_{(u,v)}\}$\;
		\For{$w \in N_{u}$}{
			\If{$(v,w) \in E_{o}$ \textbf{and} $\triangle_{uvw} \notin visited$}{
				$visited \gets visited \bigcup \triangle_{uvw}$\;
				$\tau_{\triangle_{uvw}} = min(\tau_{(u,v)}, \tau_{(u,w)}, \tau_{(v,w)})$\;
				$V_{m} \gets V_{m} \bigcup \{(u,w), \tau_{(u,w)}\}$\;
				$V_{m} \gets V_{m} \bigcup \{(v,w), \tau_{(v,w)}\}$\;
				$E_{m} \gets E_{m} \bigcup \{((u,v),(u,w)), \tau_{\triangle_{uvw}}\}$\;
				$E_{m} \gets E_{m} \bigcup \{((u,v),(v,w)), \tau_{\triangle_{uvw}}\}$\;
				$E_{m} \gets E_{m} \bigcup \{((u,w),(v,w)), \tau_{\triangle_{uvw}}\}$\;
			}
		}
	}
	run Kruskal's algorithm on $G_m$\;
	\Return{$G_m$}
	\caption{\Inducedgraph{} construction}\label{alg:\inducedgraph{}_construction}
\end{algorithm}

The truss decomposition algorithm \cite{wang2012truss} is used to compute trussness of all edges $\{\tau_{e}, e \in E_{o}\}$ in $G_o$. Although it is possible to directly compute k-truss communities based on edge trussness with BFS traversals, such an algorithm suffers from high time complexity for redundant edge access~\cite{huang2014querying}. \autoref{alg:\inducedgraph{}_construction} uses both $G_o$ and edge trussness as inputs to construct the \inducedgraph{} $G_m$ for optimal query time. The algorithm iterates through all edges of $G_o$ and create a vertex in $G_m$ for each edge $(u,v)$ in $G_o$ with weight $\tau_{(u,v)}$. Then for each unvisited neighbor triangle $\triangle_{uvw}$ of edge $(u,v)$, the algorithm creates three edges $((u,v),(u,w))$, $((u,v),(v,w))$ and $((u,w),(v,w))$ in $G_m$ with same weight $\tau_{\triangle_{uvw}} = min(\tau_{(u,v)}, \tau_{(u,w)}, \tau_{(v,w)})$. After that, one can simply run Kruskal's algorithm to get the maximum spanning forest.

\begin{figure}[ht]
    \centering
    \includegraphics[width=\linewidth]{./figures/inducedgraph.pdf}
    \caption{An example \inducedgraph{} of the example graph in \autoref{fig:example}}
    \label{fig:\inducedgraph{}}
\end{figure}

We show an example of \inducedgraph{} in \autoref{fig:\inducedgraph{}}. We outline the \inducedgraph{} of the example graph in \autoref{fig:example} with bold lines. The rest lines are edges that are generated by \autoref{alg:\inducedgraph{}_construction} but discarded by Kruskal's algorithm. 

\note{some possible error of time complexity in~\cite{huang2014querying}}
The time and space complexity for computation of edge trussness of $G_o$ are $O(\sum_{(u,v) \in E_{o}}{min\{d_{u},d_{v}\}})$ and $O(m)$ respectively~\cite{huang2014querying}. Listing all the triangles in $G_o$ takes $O(\sum_{(u,v) \in E_{o}}{min\{d_{u},d_{v}\}})$ time and $O(\sum_{(u,v) \in E_{o}}{min\{d_{u},d_{v}\}})$ space. Finally, running Kruskal's algorithm takes $O(\sum_{(u,v) \in E_{o}}{min\{d_{u},d_{v}\}}log{m})$ time. As $G_m$ is a maximum spanning forest, so the \inducedgraph{} index takes $O(|V_m|) = O(m)$ space.

\vskip 0.1in \noindent \textbf{Query on \InducedGraph{}.} To query the k-truss communities of a query vertex $q$ in $G_o$, the algorithm iterate through adjacent edges of the vertex $q$. For each neighbor edge $(u,q)$ that is unvisited by the algorithm, it is marked as a seed edge for a new community $C_i$. Suppose the edge $(u,q)$ in $G_o$ is represented as a vertex $x$ in $G_m$, the algorithm starts a BFS/DFS from vertex $x$ in $G_m$ and only expands through edges with weight $\ge k$ to find the connected component $CC$. Then if finds the represented edge $e$ of each vertex $v \in CC$ and adds $e$ to the community $C_i$. The union of all communities $A = \bigcup C_i$ is all the k-truss communities the vertex $q$ belongs to.

\begin{Thm}
The union of all communities $\bigcup C_i$ found by \autoref{alg:\inducedgraph{}_query} is the union of all the k-truss communities containing query vertex $q$.
\label{thm:\inducedgraph{}_query}
\end{Thm}

\begin{proof}
According to \autoref{def:\inducedgraph{}}, a vertex $x$ in \inducedgraph{} $G_m$ with weight $w_{x} \le k$ means the represented edge $e$ in $G_o$ has trussness $\tau_{e} \le k$ and thus can be included in a k-truss community. An edge $(x,y)$ in \inducedgraph{} $G_m$ with weight $w_{(x,y)} \le k$ means the represented triangle $\triangle$ in $G_o$ has all three edges with trussness higher or equal to $k$ and thus the triangle is included in a k-truss community containing all three edges of it. Adjacent edges in $G_m$ means adjacent triangles in $G_o$ and connected components in $G_m$ means triangle connected components in $G_o$. So, BFS/DFS search starts with a seed vertex $x$ with weight constraint will find the maximal connected component including $x$ which representing the k-truss community that $e$ belongs to in $G_o$ ($x$ represents $e$ in $G_m$). Therefore, performing such BFS/DFS searches on each edge of the query vertex will find all the k-truss communities that the query vertex belongs to.
\end{proof}

\begin{algorithm}
	\KwData{$G_{o}(V_{o},E_{o})$, $G_{m}(V_{m},E_{m})$, an integer $k$, a query vertex $q$}
	\KwResult{a union of all k-truss communities $\bigcup C_i$ containing $q$}
	\BlankLine
	$i \gets 0$, $visited \gets \emptyset$\;
	\For{$u \in N_{q}$}{
		\If{$(u,q) \notin visited$}{
			find representing vertex $x$ of $(u,q)$ in $G_m$\;
			$CC \gets$ connected component containing $x$ with edges of weight $\ge k$\;
			$C_i \gets \emptyset$\;
			\For{$v \in CC$}{
				find represented $e$ of $v$ in $G_o$\;
				$visited \gets visited \bigcup e$\;
				$C_i \gets C_i \bigcup e$\;
			}
			$i \gets i + 1$\;
		}
	}
	\Return{$\bigcup C_i$}
	\caption{Query on \inducedgraph{}}\label{alg:\inducedgraph{}_query}
\end{algorithm}

Since the query process is performing a BFS on a maximum spanning forest, each query takes $O(|A|)$ time and $O(|A|)$ space, where $|A|$ is the number of edges in $A$. Although such time complexity is already optimal if the detailed communities are required. We propose a new index structure that can be constructed upon the \inducedgraph{} to further reduce the time complexity if details of k-truss communities are not required.

\subsection{\TreeIndex{}}

We first show how to construct the \treeindex{} based on \inducedgraph{}. Then we design an algorithm to efficiently query the \treeindex{}. 

\vskip 0.1in \noindent \textbf{\TreeIndex{} Construction.} A key observation in~\cite{cohen2008trusses} is that, for $k \ge 2$, each k-truss of $G_o$ is the subgraph of a (k-1)-truss of $G_o$. With this observation, for k-truss communities, we have the following theorem.

\begin{Thm}
A k-truss community $C_k$ is the subgraph of a l-truss community $C_l$, if $C_k$ and $C_l$ are triangle connected and $l < k$. If k-truss community $C_k$ is the subgraph of both $l_1$-truss community $C_{l_1}$ and $l_2$-truss community $C_{l_2}$, then $l_1 \neq l_2$.
\label{thm:truss_hierarchy}
\end{Thm}

\begin{proof}
For the first part, since $l < k$, if edges in $C_k$ are triangle connected through triangles with trussness of $k$, then they are also triangle connected through triangles with trussness of $l$. 

\note{do we call it k-truss or $l_1$-truss.}
For the second part, suppose $l_1 = l_2$, then edges in $C_{l_1}$ and $C_{l_2}$ are triangle connected through $C_k$. So $C_{l_1} \bigcup C_{l_2}$ meets the definition of k-truss community (\autoref{def:k-truss_community}) and becomes a larger k-truss community. This contradicts with $C_{l_1}$ and $C_{l_2}$ are k-truss communities themselves, \ie they are maximal k-truss.
\end{proof}

\begin{algorithm}
	\KwData{$G_{m}(V_{m},E_{m})$}
	\KwResult{$G_{t}(V_{t},E_{t})$, $h$}
	\BlankLine
	$Q \gets \emptyset$, $parent \gets \emptyset$\;
	\While{$V_{m} \neq \emptyset$}{
		$seed \gets$ an unvisited vertex in $V_{m}$, $Q \gets Q \bigcup seed$\;
		\While{$Q \neq \emptyset$}{
			$x = Q.pop()$\;
			\For{$z \in N_x$}{
				$Q \gets Q \bigcup z$, $parent[z] \gets x$\;
			}
			\eIf{$x \in parent$}{
				$y \gets parent[x]$, $C_a \gets C_{y}^{max}$\;
				\While{$\tau_{C_a} > w_{(x,y)}$}{
					$C_c \gets C_a$, $C_a \gets$ parent of $C_a$ in $G_t$\;
					\If {$C_a = \emptyset$}{
						$\tau_{C_a} \gets -1$ \Comment{Reach the top of the tree.}\;
					}
				}
				\eIf{$\tau_{C_a} < w_{(x,y)}$}{
					\eIf{$w_{(x,y)} = w_{x}$}{
						create $C_{x}^{max}$, $h[x] \gets C_{x}^{max}$\;
						$C_{x}^{max}.parent \gets C_a$, $C_{c}.parent \gets C_{x}^{max}$\;
					}{
						create $C_{x}^{max}$, $h[x] \gets C_{x}^{max}$\;
						create $C_{(x,y)}$, $C_{(x,y)}.parent \gets C_a$\;
						$C_{c}.parent \gets C_{(x,y)}$\;
						$C_{x}^{max}.parent \gets C_{(x,y)}$\;
					}
				}{
					\eIf{$w_{(x,y)} = w_{x}$}{
						$h[x] \gets C_a$\;
					}{
						create $C_{x}^{max}$, $h[x] \gets C_{x}^{max}$\;
						$C_{x}^{max}.parent \gets C_a$\;
					}
				}
			}{
				create $C_{x}^{max}$, $h[x] \gets C_{x}^{max}$\;
				$V_{t} \gets V_{t} \bigcup C_{x}^{max}$\;
			}
			remove $x$ from $V_{m}$\;
		}
	}
	\Return{$G_{t}(V_{t},E_{t})$, $h$}
	\caption{\Treeindex{} Construction}\label{alg:\treeindex{}_construction}
\end{algorithm}

According to \autoref{thm:truss_hierarchy}, we can build another tree-structured index upon our existing \inducedgraph{} to further facilitate KTruss computation. In this new tree-structured index, we use vertices to represent k-truss communities, \ie we assign each k-truss community an unique ID and a representing vertex in the new index. If one k-truss community is the subgraqh of another k-truss community, we assign an edge to connect the representing vertices. Each vertex can have a list associated with it including the status of the related k-truss community, such as the trussness of the community, the size of the community, etc. We call this new index the \treeindex{} and denote it as $G_t$. For each vetx of $G_t$, we also have meta data of the represented k-truss communities, \eg the trussness, the size, etc., stored with it. These meta-data can be gathered very easily through the index construction process. For the ease of query, we build a hash table $h$ that for each edge $e$ in $G_o$ (vertex $x$ in $G_m$), we record the ID of the k-truss community that includes it with highest order $k$. We denote such a k-truss community as $C^{max}$. We have the following theorem for $G_t$.

\begin{Thm}
The \treeindex{} $G_t$ is a forest.
\label{thm:forest}
\end{Thm}

\begin{proof}
First, according to \autoref{thm:truss_hierarchy}, there is only one ancestor for each k-truss community for . Also, there is no inter level edges according to the definition of maximal KTruss. So, if the graph contains a loop, then a KTruss may contains more than 1 ancestors.

Second, $G_t$ can be disconnected as not all k-truss communities are triangle connected with each other. 
\end{proof}

\autoref{alg:\treeindex{}_construction} shows the procedure to build the \treeindex{} $G_t$. The algorithm uses BFS to traverse the \inducedgraph $G_m$. For each vertex $x$, if it does not have a parent vertex in the BFS traversal, then the algorithm uses it as a seed vertex to create a new index tree. Otherwise it is combined to the same index tree $T \in G_t$ as its parent vertex $y$. According to \autoref{thm:\inducedgraph{}_vertex_trussness}, we have the following equation.

\begin{equation}
w_{x} \ge w_{(x,y)}, w_{y} \ge w_{(x,y)}
\label{equ:lowest_edge_weight}
\end{equation}

\begin{Thm}
For a vertex $x$ and its neighbor vertex $y$ in \inducedgraph{} $G_m$, if their representing edges in $G_o$ are contained in the same k-truss community with trussness of $k$, then $k \le w_{(x,y)}$. 
\label{thm:k_le_edge_weight}
\end{Thm}

\begin{proof}
Since $G_m$ is the maximum spanning forest, it has the cycle property, \ie for any cycle in $G_{m}^{\prime}$, if the weight of an edge in the cycle is smaller than the individual weights of all the other edges in the cycle, then this edge cannot belong to a maximum spanning forest. So there is no path in $G_{m}^{\prime}$ between $x$ and $y$ that has all edges with weight $> w_{(x,y)}$. Suppose $x$ and $y$ representing $e_x$ and $e_y$ in $G_o$, this means that $e_x$ is not triangle connected to $e_y$ through edges in $G_o$ with trussness $> w_{(x,y)}$. Therefore, it is not possible for $e_x$ and $e_y$ to exist in the same k-truss community with $k > w_{(x,y)}$.
\end{proof}
 
Having a parent $y$ in the BFS search only means that the vertex $x$ can be combined to the current index tree $T$. We still have a problem to solve: On which part of $T$ should the algorithm add the vertex $x$? According to \autoref{thm:k_le_edge_weight} and \autoref{equ:lowest_edge_weight}, the algorithm needs to backtrack $T$ from $C_{y}^{max}$ to find an ancestor vertex $C_a$ that meets $\tau_{C_a} \le w_{(x,y)}$ and use it as the merge point of $x$. We refer to the index vertices $C_{y}^{max},...,C_{i},...,C_{a}$ as the backtrack branch for vertex $x$ in $T$ and denote it as $B$. % If no such an index vertex $C_a$ can be found, \ie the algorithm reaches the root of $T$ and has not found a qualified ancestor vertex, it creates a new index vertex $C_{x}^{max}$ with trussness $\tau_{C_{x}^{max}} = w_{x}$ and uses it as the new root of $T$, \ie the algorithm attaches the old root of $T$ as a child vertex to $C_{x}^{max}$. \note{this part need to modify}

Once the algorithm has found $C_a$, it needs to check the relations of $\tau_{C_a}$, $w_{(x,y)}$ and $w_{x}$ to decide how to merge vertex $x$ to $T$. Note that they follow $\tau_{C_a} \le w_{(x,y)} \le w_{x}$, so we have 4 cases shown in \autoref{alg:\treeindex{}_construction}. As long as $\tau_{C_a} \neq w_{x}$, we create a new index vertex $C_{x}^{max}$ with trussness $\tau_{C_{x}^{max}} = w_{x}$. If $\tau_{C_a} < w_{(x,y)} < w_{x}$, we also create a new index vertex $C_{(x,y)}$ with trussness $\tau_{C_{(x,y)}} = w_{(x,y)}$. Then we adjust the tree structure of $T$ with new index vertices. Finally, we update the hash table to record in which index vertex $x$ is.
% If there is a child index vertex of $C_a$ on backtrack branch $B$, we denote it as $C_c$, and we have $\tau_{C_c} > w_{(x,y)}$. We have the following 4 cases:

%\begin{itemize}
	%\item $\tau_{C_a} < w_{(x,y)}$
	%\begin{itemize}
		%\item $w_{(x,y)} = w_{x}$: \\
		%Create a new index vertex $C_{x}^{max}$ with trussness $\tau_{C_{x}^{max}} = w_{x}$, and attach it to $T$ as a child vertex of $C_a$. Attach $C_c$ to $C_{x}^{max}$ as a child vertex since $\tau_{C_{x}^{max}} = w_{x} = w_{(x,y)} < \tau_{C_c}$. Update the hash table $h[x] = C_{x}^{max}$.
		%\item $w_{(x,y)} < w_{x}$: \\
		%Create a new index vertex $C_{x}^{max}$ with trussness $\tau_{C_{x}^{max}} = w_{x}$. Create another index vertex $C_{(x,y)}$ with trussness $\tau_{C_{(x,y)}} = w_{(x,y)}$. Attach the index vertex $C_{(x,y)}$ to $C_a$ as a child and attach both $C_{x}^{max}$ and $C_c$ as two separate children to the index vertex $C_{(x,y)}$. Update the hash table $h[x] = C_{x}^{max}$.
	%\end{itemize}
	%\item $\tau_{C_a} = w_{(x,y)}$
	%\begin{itemize}
		%\item $w_{(x,y)} = w_{x}$: \\
		%Update the hash table $h[x] = C_{a}$.
		%\item $w_{(x,y)} < w_{x}$: \\
		%Create a new index vertex $C_{x}^{max}$ with trussness $\tau_{C_{x}^{max}} = w_{x}$. Attach it to $T$ as a separate child to $C_a$ along side with $C_c$. Update the hash table $h[x] = C_{x}^{max}$.
	%\end{itemize}
%\end{itemize}

For each vertex of $G_m$, the backtrack procedure takes $O(k_{max})$ time, where $k_{max}$ is the highest trussness of any k-truss community in $G_o$. Since the index construction process is a BFS on a maximum spanning tree, the \treeindex{} construction algorithm takes $O(k_{max}m)$ time. As each vertex in $G_t$ represents a k-truss community in $G_o$, and $G_t$ is a forest. The algorithm takes $O(m)$ space and the index size is also $O(m)$ space. Although in practice, the size of $G_t$ is much smaller than $O(m)$.

%\begin{Thm}
%When $\tau{A} < \tau{e}$, the vertex $u$ only connects to this branch of index vertex $A$ with currently discovered edges, not other branches.
%\end{Thm}
%
%\begin{proof}
%If it want connect to other branches, it need to connect through ancestors. But $A$ has a lower trussness that can not connect higher k truss communities.
%\end{proof}

\vskip 0.1in \noindent \textbf{Query on \TreeIndex{}.}
\Treeindex{} supports three basic types of k-truss community queries of a single query vertex $q$ as listed below.

\begin{itemize}
	\item{K-truss query:} Given a vertex $q$ and an integer $k$, find the k-truss community that contains $q$.
	\item{Max-k-truss query:} Given a vertex $q$, find the k-truss community with highest possible trussness that contains $q$.
	\item{Any-k-truss query:} Given a vertex $q$, find all the k-truss communities that contains $q$.
\end{itemize}

Max-k-truss query is naturally supported by simply looking up the hash table $h$ and comparing trussness of $h[x_e]$ for each neighbor edge. We show the queries process algorithms for k-truss query and any-k-truss query in \autoref{alg:\treeindex{}_query}. A common operation used in both query algorithms is what we called backtrack branch search, which is defined in \autoref{def:backtrack_branch_search} below. We can see that if a specific $k$ is provided, the backtrack branch search will stop once the trussness falls below $k$. On the other hand, if no $k$ is provided, a value of $0$ is used and the search will reach the root of the tree.

\begin{Def}[Backtrack branch search]
Given a vertex $C_{0} \in G_t$ and an integer $k$, the backtrack branch search returns a list of vertices $C_{0},...,C_{i},...$ that $C_{i+1}$ is the parent vertex of $C_{i}$ in $G_t$ and any vertex $C_{i}$ meets $\tau_{C_{i}} \ge k$. We refer to the searching results $C_{0},...,C_{i},...$ as backtrack branch and denote it as $B$.
\label{def:backtrack_branch_search}
\end{Def}

\Treeindex{} also supports all three types of queries when the input is a set of query vertices $Q$. The query process algorithms simply takes intersections of the query results of each individual query vertex for k-truss queries and any-k-truss queries. For max-k-truss queries, the query process algorithm needs to calculate the least common ancestors in $G_t$ of the results of each individual query vertex.

\begin{algorithm}
	\KwData{$G_{o}(V_{o},E_{o})$, $G_{t}(V_{t},E_{t})$, the hash table $h$, a query vertex $q$ or a set of query vertices $Q$, [an integer k]}
	\KwResult{a set of k-truss community IDs $R$}
	\SetKwProg{Fn}{function}{}{end}
	\BlankLine
	\Fn{branch\_search ($C \in G_t$, $G_t$, [$k$ = 0])}{
		$B \gets \emptyset$\;
		\While{$C \neq \emptyset$ \textbf{and} $\tau_{C} \ge k$}{
			$B \gets B \bigcup {C}$\;
			$C \gets$ parent of $C$ in $G_t$\;
		}
		\Return{$B$}
	}
	\BlankLine
	\Fn{query\_k ($q$, $G_o$, $G_t$, $k$)}{
		$R \gets \emptyset$\;
		\For{$e \in N_q$}{
			$B$ = branch\_search ($h[x_e]$, $G_t$, $k$)\;
			\If{$\tau_{B[-1]} = k$}{
				$R \gets R \bigcup B[-1]$; \Comment{$B[-1]$ is the last element in $B$}
			}
		}
		\Return{$R$}
	}
	\BlankLine
	\Fn{query\_anyk ($q$, $G_o$, $G_t$)}{
		$R \gets \emptyset$\;
		\For{$e \in N_q$}{
			$B$ = branch\_search ($h[x_e]$, $G_t$)\;
			$R \gets R \bigcup B$\;
		}
		\Return{$R$}
	}
	\caption{Query on \Treeindex{}}\label{alg:\treeindex{}_query}
\end{algorithm}

For single vertex queries, the time complexity is $O(d_q)$ for max-k-truss queries and $O(\sum_{e \in N_q}\tau_{h[x_e]})$ for k-truss and any-k-truss queries. The space complexity is $O(1)$ for max-k-truss queries, $O(d_q)$ for k-truss queries and $\sum_{e \in N_q}\tau_{h[x_e]}$ for any-k-truss queries. For multiple vertices max-k-ktruss queries, since the least common ancestor computation takes $O(H)$\footnote{$O(H)$ is for simple online algorithm, off-line algorithms can achieve time complexity of $O(1)$~\cite{bender2000lca}.} time, where $H$ is the height of the tree. The query time is $O(\sum_{q \in Q}(\max_{e \in N_q}\tau_{h[x_e]} + d_q))$ and the space is $O(\max_{q \in Q}\max_{e \in N_q}\tau_{h[x_e]})$. Multiple vertices k-truss queries take $O(\sum_{q \in Q}\sum_{e \in N_q}\tau_{h[x_e]})$ time and $O(\max_{q \in Q}d_q)$ space. For multiple vertices any-k-truss queries, the time and space complexity is $O(\sum_{q \in Q}\sum_{e \in N_q}\tau_{h[x_e]})$ and $O(\max_{q \in Q}\sum_{e \in N_q}\tau_{h[x_e]})$ respectively.

%\Treeindex{} supports multiple types of k-truss community queries. We start by defining backtrack branch search and least common ancestor. Then we introduce four most common types of queries supported by \treeindex{}.
%
%\begin{Def}[Backtrack branch search]
%Given a vertex $C_{0} \in G_t$ and an integer $k$, the backtrack branch search returns a list of vertices $C_{0},...,C_{i},...$ that $C_{i+1}$ is the parent vertex of $C_{i}$ in $G_t$ and any vertex $C_{i}$ meets $\tau_{C_{i}} \ge k$. We refer to the searching results $C_{0},...,C_{i},...$ as backtrack branch and denote it as $B$.
%\label{def:backtrack_branch_search}
%\end{Def}
%
%\begin{Def}[Least common ancestor]
%Given a set of vertices $C_{0},...,C_{s}$ in $G_t$, if $C_{0},...,C_{s}$ belong to the same tree $T$, the least common ancestor of $C_{0},...,C_{s}$ is the vertex furthest from the root of $T$ that is an ancestor of all vertices in $C_{0},...,C_{s}$. We denote the least common ancestor as the $LCA$.
%\label{def:lca_search}
%\end{Def}
%
%\Treeindex{} supports both single vertex query and multiple vertices query. It also support query for k-truss communities of a specific $k$ or any possible $k$. Here we list four most common types of queries supported by \treeindex{}.
%
%\begin{itemize}
	%\item \textbf{Single vertex and a specified $k$.} \\
	%Given a query vertex $q$ and an integer $k$, find the k-truss community that contains $q$ with trussness $k$. The query algorithm iterates neighbor edges of $q$. For each edge $e$, let $x_e$ be the representing vertex in $G_m$. The algorithm performs a backtrack branch search with $h[x_e]$ and $k$ in $G_t$ and get the backtrack branch $B_e$. Let $C_e$ contain the last vertex in $B_e$ if it has trussness equals $k$ or be $\emptyset$ otherwise. The union of results of all neighbor edges $\bigcup_{e} C_e$ is all the k-truss communities that contains $q$ with trussness of $k$. 
	%\item \textbf{Single vertex with any possible $k$.} \\
	%Given a query vertex $q$, find all the k-truss communities that contains $q$. Similar to last one, the query algorithm also performs a backtrack branch search on each neighbor edge of $q$. The union of results of all neighbor edges $\bigcup_{e} B_e$ are all the k-truss communities that contains $q$. 
	%\item \textbf{Multiple vertex with a specified $k$.} \\
	%Given a set of query vertices $Q$, find the k-truss community that contains all vertices in $Q$ with trussness $k$. For each $q \in Q$, find $R_q = \bigcup_{e} C_e$ using the single vertex query algorithm. The intersection of results of all query vertices $\bigcap_{q} R_q$ is all the k-truss communities that contains all vertices in $Q$ with trussness of $k$.
	%\item \textbf{Multiple vertex with any possible $k$.} \\
	%Given a set of query vertices $Q$, find all k-truss communities that contains all vertices in $Q$. For each $q \in Q$, let $e_q$ be neighbor edges of $q$ and $x_{e_q}$ be the representing vertex in $G_m$. We denote $\bigcup_{e_q} h[x_{e_q}]$ as $H_q$. For two vertices $p,q \in Q$, the query algorithm combines $H_p$ and $H_q$ in the following way. For each pair of vertices $h[x_{e_p}] \in H_p$ and $h[x_{e_q}] \in H_q$, the algorithm finds the least common ancestor of $h[x_{e_p}]$ and $h[x_{e_q}]$ in $G_t$, if there is any. We denote it as $LCA_{\{e_{p},e_{q}\}}$. The union of least common ancestor of each pair of vertices $H_{\{p,q\}} = \bigcup_{\{e_{p},e_{q}\}} LCA_{\{e_{p},e_{q}\}}$ is the combining result of $H_p$ and $H_q$. The algorithm iteratively combines all query vertices in this way and gets $H_Q$. Then for each index vertex in $H_Q$, the algorithm performs a backtrack branch search to and return the union of the results $R_Q$ which is all the k-truss communities that contains all vertices in $Q$. 
%\end{itemize}

%\begin{algorithm}
	%\KwData{$G_{o}(V_{o},E_{o})$, $G_{t}(V_{t},E_{t})$, the hash table $h$, a query vertex $q$ or a set of query vertices $Q$, [an integer k]}
	%\KwResult{a set of k-truss community IDs $R$}
	%\SetKwProg{Fn}{function}{}{end}
	%\BlankLine
	%\Fn{branch\_search ($C \in G_t$, $G_t$, [$k$ = 0])}{
		%$B \gets \emptyset$\;
		%\While{$C \neq \emptyset$ \textbf{and} $\tau_{C} \ge k$}{
			%$B \gets B \bigcup {C}$\;
			%$C \gets$ parent of $C$ in $G_t$\;
		%}
		%\Return{$B$}
	%}
	%\BlankLine
	%\Fn{query\_singleq ($q$, $G_o$, $G_t$, $k$)}{
		%$R \gets \emptyset$\;
		%\For{$e \in N_q$}{
			%$B$ = branch\_search ($h[x_e]$, $G_t$, $k$)\;
			%\If{$\tau_{B[-1]} = k$}{
				%$R \gets R \bigcup B[-1]$; \Comment{$B[-1]$ is the last element in $B$}
			%}
		%}
		%\Return{$R$}
	%}
	%\BlankLine
	%\Fn{query\_singleq\_anyk ($q$, $G_o$, $G_t$)}{
		%$R \gets \emptyset$\;
		%\For{$e \in N_q$}{
			%$B$ = branch\_search ($h[x_e]$, $G_t$)\;
			%$R \gets R \bigcup B$\;
		%}
		%\Return{$R$}
	%}
	%\BlankLine
	%\Fn{query\_multiq ($Q$, $G_o$, $G_t$, $k$)}{
		%$R \gets \emptyset$, $initialized \gets False$\;
		%\For{$q \in Q$}{
			%\eIf{$initialized = False$}{
				%$R \gets$ query\_singleq ($q$, $G_o$, $G_t$, $k$)\;
				%$initialized \gets True$\;
			%}{
				%$R \gets R \bigcap$ query\_singleq ($q$, $G_o$, $G_t$, $k$)\;
			%}
		%}
		%\Return{$R$}
	%}
	%\BlankLine
	%\Fn{query\_multiq\_anyk ($Q$, $G_o$, $G_t$)}{
		%$R \gets \emptyset$, $initialized \gets False$\;
		%\For{$q \in Q$}{
			%\eIf{$initialized = False$}{
				%$R \gets$ query\_singleq\_anyk ($q$, $G_o$, $G_t$)\;
				%$initialized \gets True$\;
			%}{
				%$R^{\prime} \gets$ query\_singleq\_anyk ($q$, $G_o$, $G_t$)\;
			%}
		%}
		%\Return{$R$}
	%}
	%\caption{Query on \inducedgraph{}}\label{alg:\inducedgraph{}_query}
%\end{algorithm}
%
%For single vertex queries, the time complexity is $O(\sum_{e \in N_q}\tau_{h[x_e]})$, the space complexity is $O(d_q)$ if $k$ is specified or $\sum_{e \in N_q}\tau_{h[x_e]}$ otherwise. For multiple vertices queries, when $k$ is specified, the time complexity is $O(\sum_{q \in Q}\sum_{e \in N_q}\tau_{h[x_e]})$ and the space complexity is $O(\sum_{q \in Q}d_q)$. Since the least common ancestor computation takes $O(h)$\footnote{$O(h)$ is for simple online algorithm, off-line algorithms can achieve time complexity of $O(1)$~\cite{bender2000lca}.} time, where $h$ is the height of the tree. If $k$ is not specified, the time complexity is $O(\sum_{q \in Q}\sum_{e \in N_q}\tau_{h[x_e]})$ and the space complexity is $O(\sum_{q \in Q}\sum_{e \in N_q}\tau_{h[x_e]})$.

\subsection{Query on \TwoLevelIndex{}}
\label{query}
We classify k-truss local community queries into two categories according to the level of information required. The \toplevelprob{} query (Section~\ref{\toplevelprob{}}) requires only information of relations between k-truss communities and to which k-truss communities query vertices belong. For example, "Do query vertices belongs to same k-truss communities?". This type of queries can be answered solely by the top level index of our \twolevelindex{}. Another type of queries, which requires edge level information to process, is called the \bottomlevelprob{} query (Section~\ref{\bottomlevelprob{}}). For example, the widely studied community search queries. We process this type of queries by first locating the target k-truss communities with the top level index and then diving into the edge-level details with the bottom level index. Besides of the two query categories, we also incorporate the ability to add various cohesiveness criteria for queries with our \twolevelindex{}.

\subsubsection{The \TopLevelProb{} Query}
\label{\toplevelprob{}}
%K-truss community info queries have the coarsest granularity, \ie it only ask for information on the community level and don't require any details inside a k-truss community.  Queries like these can be directly answered by only looking up in the top level \treeindex{} of the \twolevelindex{}. Since each vertex contains meta-data that stores properties of the corresponding k-truss community, this type of query can also answer queries about the trussness of communities, the size of communities, etc.

~\\The \twolevelindex{} supports any range of trussness value as cohesiveness criterion for a query. They all support both a single query vertex and a set of query vertices. There are three most common cohesiveness creteria: a specific $k$ value, the maximum $k$ value and any $k$ values. We refer to them as k-truss queries, max-k-truss queries, and any-k-truss queries. 

\begin{algorithm}
	\KwData{$G^{o}(V^{o},E^{o})$, $G^{c}(V^{c},E^{c})$, $H$, $Q$}
	\KwResult{subgraph $S$ of $G^c$}
	\SetKwProg{Fn}{function}{}{end}
	\BlankLine
	$S \gets \emptyset$\;
	$initialized \gets false$\;
	\For{$u^o \in Q$}{
		$SS \gets$ singlev\_subgraph($u^o$)\;
		\lIf{$!init$} {
			$S \gets SS$, $init \gets true$
		}
		\lElse {
			$S \gets S \cap SS$
		}
	}
	\Return{$S$}
	\BlankLine
	%\Fn{branch\_search ($u^s$)}{
		%$B \gets \emptyset$\;
		%\While{$u^c \neq null$} {
			%$B \gets B \bigcup u^s$\;
			%$u^c \gets u^{c}.parent$\;
		%}
		%\Return{$B$}
	%}
	%\BlankLine
	\Fn{singlev\_subgraph ($u^o$)}{
		$SS \gets \emptyset$\;
		\For{$v^o \in N_u$} {
			$u^c \gets H[(u^o, v^o)]$\;
			$B \gets$ ancestors of $u^c$ in $G^c$\;
			$SS \gets SS \bigcup B$\;
		}
		\Return{$SS$}
	}
	\caption{$union-intersection$ Algorithm.}\label{alg:union_intersection}
\end{algorithm}

\toplevelprob{} k-truss community queries share a similar querying process on the top level index, called $union-intersection$ procedure. We show the detailed procedure in Algorithm \ref{alg:union_intersection}. Given the \twolevelindex{} and a lookup table $H$ that maps represents edges in the original graph $G^o$ (represented by vertices in $G^m$) to vertices in the \treeindex{} $G^c$ as input, the algorithm first iterate through adjacent edges of each query vertex. For each edge maps to a vertex in $G^c$, the the algorithm takes the $union$ of it and its ancestors in $G^c$, to which represent all the communities a query vertex belongs. Then the algorithm take the $intersection$ of the results of all query vertices, to which represents communities that all query vertices belong. Finally, we can easily retrieve communities that have weights of a specified $k$ (k-truss query), calculate communities that have maximum trussness (Max-k-truss query) or return all the vertices (Any-k-truss query).  

%\begin{Thm}
%The union of all communities $\bigcup C_i$ found by \autoref{alg:\inducedgraph{}_query} is the union of all the k-truss communities containing query vertex $q$.
%\label{thm:\inducedgraph{}_query}
%\end{Thm}
%
%\begin{proof}
%According to \autoref{def:\inducedgraph{}}, a vertex $x$ in \inducedgraph{} $G_m$ with weight $w_{x} \le k$ means the represented edge $e$ in $G_o$ has trussness $\tau_{e} \le k$ and thus can be included in a k-truss community. An edge $(x,y)$ in \inducedgraph{} $G_m$ with weight $w_{(x,y)} \le k$ means the represented triangle $\triangle$ in $G_o$ has all three edges with trussness higher or equal to $k$ and thus the triangle is included in a k-truss community containing all three edges of it. Adjacent edges in $G_m$ means adjacent triangles in $G_o$ and connected components in $G_m$ means triangle connected components in $G_o$. So, BFS/DFS search starts with a seed vertex $x$ with weight constraint will find the maximal connected component including $x$ which representing the k-truss community that $e$ belongs to in $G_o$ ($x$ represents $e$ in $G_m$). Therefore, performing such BFS/DFS searches on each edge of the query vertex will find all the k-truss communities that the query vertex belongs to.
%\end{proof}

The time and space complexity for collecting ancestor for a given super-vertex is $\tau_e$. To iterate all the adjacent edges of a query vertex takes $\sum_{v \in N_u}{\tau_{(u,v)}}$ time and space. Finally, the algorithm needs to find the set of super-vertex for each query vertex to get the set of common super-vertex, so the total time and space complexity for the $union-intersection$ procedure is $\sum_{u \in Q}{\sum_{v \in N_u}{\tau_{(u,v)}}}$. 
%It is the time and space complexity for all three types of k-truss community info queries and they all can be answered by checking the results one pass.

\subsubsection{The \BottomLevelProb{} Query}
\label{\bottomlevelprob{}}

~\\The \bottomlevelprob{} k-truss community query requires information of finest granularity as it needs to explore the inner edge-level structure of a k-truss community. Our bottom level index contains the detailed triangle connectivity information that makes such queries possible. To process a k-truss community query, we first locate the target k-truss communities with the top level index and then compute query results using edge-level details provided by the bottom level index. 
%We show three concrete examples of this type of queries in this section: the k-truss community search (Section~\ref{k-truss community search}), the k-truss community boundary search (Section~\ref{boundary search}) and the triangle connected maximin path search (Section~\ref{path search}).

We use k-truss community search as a concrete example as it is the simplest form of the \bottomlevelprob{} k-truss community query. First, the $union-intersection$ algorithm is performed to get target communities of the query. Then for each community in target communities, we collect edges contained in it by gathering the vertex list of subgraphs of $G^m$ stored alongside $G^c$ vertices. Finally, edges of the original graph $G^o$ can be retrieved by converting their corresponding vertices in $G^m$. 
%As we mentioned in section \ref{structure}, the partial $G^m$ is stored as adjacent list. We only need to collect all the vertices of the partial $G^m$, which represent edges in the orignal graph $G^o$. 

%\begin{algorithm}
	%\KwData{$G^{o}(V^{o},E^{o})$, $G^{m}(V^{m},E^{m})$, $G^{c}(V^{c},E^{c})$, $H$, $Q$}
	%\KwResult{all k-truss communities $C$ containing $Q$}
	%\BlankLine
	%$S \gets union-intersect$($G^o$, $G^c$, $H$, $Q$)\;
	%\For{$v^c \in S$} {
		%$C_i \gets \emptyset$\;
		%\For{$v^m \in v^{c}.adj\_list.keys$}{
			%find corresponding $e^o$ of $v^m$;
			%$C_i \gets C_i \bigcup e^o$\;
		%}
	%}
	%\Return{$\bigcup{C_i}$}
	%\caption{K-truss Community Search Query}\label{alg:search_query}
%\end{algorithm}

The $union-intersect$ algorithm takes $\sum_{u \in Q}{\sum_{v \in N_u}{\tau_{(u,v)}}}$ time and space. Each edge in the target communities will only be accessed exactly once, so the time and space complexity for the search are $\sum_{u \in Q}{\sum_{v \in N_u}{\tau_{(u,v)}}} + |\bigcup{C_i}|$, where $\bigcup{C_i}$ is the union of target communities.

%The intra-community query requires information of finest granularity as it needs to explore the inner structure of a k-truss community. Our bottom level index, \inducedgraph{}, contains the detailed triangle connectivity information that make such queries possible. For some queries, the top level \treeindex{} may also help. We show two concrete examples of such queries in this section: finding boundaries among k-truss communities and finding a triangle connected maximin trussness path between two query vertices, etc. % For the sake of space, we omitted the detailed algorithm here.

%\vskip 0.1in \noindent \textbf{k-truss community boundary search}
%%\label{boundary search}
%
%The boundary of a k-truss community $C_i$ contains edges in the community that have triangle adjacent neighbor edges belong to other k-truss communities $C_j, C_j \notin C_i$. To find the boundary of a k-truss community, as vertices in the \treeindex{} $G^c$ have subgraphs of the \inducedgraph{} $G^m$ stored in it, we can first gather the subgraph $S$ of $G^o$ that contains children vertices of the corresponding vertex $v^c$ of the queried k-truss community $C$ using the top level index. Then we iterate through all the vertices of the subgraph of $G^m$ that stored in $S$ and select vertices that have neighbors stored in other $G^c$ vertices $u^c$ that $u^c \notin S$. Finally, the collected vertices of $G^m$ can be mapped back to edges of the original graph $G^o$. Since the search exams all the edges in the queried community, the algorithm's time complexity is $O(|C|)$, and the space complexity is $O(|B|)$ , where $B$ denotes the returned boundary. For the sake of space, we omitted the detailed algorithm here.
%
%\vskip 0.1in \noindent \textbf{Triangle connected maximin path search}
%%\label{path search}
%
%Given two query vertices, a triangle connected maximin path is a path connecting two queries vertices that has all the edges are triangle connected and maximizes the minimum edge trussness. To find such a path, the algorithm first perform a max-k-truss \toplevelprob{} query to find communities containing both query vertices that have the highest trussness, denoted by $C_{max\tau}$. Then in one of the target communities, the algorithm starts a breadth first search that amid to find a path connecting any pair of edges of the source and target vertices. Due to the property of a maximum spanning tree, the found path is a maximin path of edge trussness. Such a path is guaranteed to exist as long as a max-k-truss community containing both the source and target vertices can be found because edges belonged to the same k-truss community is triangle connected. The algorithm performs a BFS after a max-k-truss info query, so the time complexity is $\sum_{v \in N_{src}}{\tau_{(src,v)}} + \sum_{v \in N_{dst}}{\tau_{(dst,v)}} + \min_{C \in C_{max\tau}}{|C|}$ and space complexity is $O(|P|)$ , where $P$ denotes the returned path. For the sake of space, we omitted the detailed algorithm here.
%%starts a breadth first search at a vertex in the stored partial $G^m$ that has the highest degree. The search ends when it reach both a vertex in $G^m$ representing an adjacent edge $e_{src}$ of the source vertex in $G^o$ and a vertex in $G^m$ representing an adjacent edge $e_{dst}$ of the target vertex $v_{dst}$. The algorithms performs a BFS after a max-k-truss info query, so the time complexity is $\sum_{v \in N_{src}}{\tau_{(src,v)}} + \sum_{v \in N_{dst}}{\tau_{(dst,v)}} + \min_{C \in C_{max\tau}}{|C|}$.

\section{Evaluations}
\label{evaluation}

In this section, we evaluate our proposed index structure for various types of k-truss community related queries on real-world networks. We first compare the \twolevelindex{} with state-of-the-art solutions, the TCP index (\cite{huang2014querying}) and the Equitruss index (\cite{akbas2017truss}) for index construction (Section~\ref{eval_const}) and single vertex k-truss community search (Section~\ref{eval_singlev_k_compare}). Then we show the effectiveness of our index for all three types of \toplevelprob{} k-truss community queries and their corresponding k-truss community search with single and multiple query vertices (Section~\ref{eval_top_bottom_compare} \& Section~\ref{eval_k_type_compare}). Finally, we analyze results of \bottomlevelprob{} k-truss community queries (Section~\ref{eval_bottom_analysis}). All experiments are implemented in C++ and are run on a Cloudlab\footnote{www.cloudlab.us} c8220 server with 2.2GHz CPUs and 256GB memory. 
%\subsection{Datasets}
\vskip 0.1in \noindent \textbf{Datasets} 

\begin{table}
\caption{Datasets}
\label{table:datasets} 
\begin{threeparttable}
	\centering
		\begin{tabularx}{\linewidth}{c|*{5}{Y}} 
		\toprule
			Dataset & Type & $|V_{wcc}|$ & $|E_{wcc}|$ & $|{\triangle}_{wcc}|$ & $k_{max}$ \\
			\midrule
			Wiki & Comm. & 2.4M & 4.7M & 9.2M & 53 \\ 
			Baidu & Web & 2.1M & 17.0M & 25.2M & 31 \\
			Skitter & Internet & 1.7M & 11.1M & 28.8M & 68 \\ 
			Sinaweibo & Social & 58.7M & 261.3M & 213.0M & 80 \\ 
			Livejournal & Social & 4.8M & 42.8M & 285.7M & 362 \\ 
			Orkut & Social & 3.1M & 117.2M & 627.6M & 78 \\
			Bio & biological & 42.9K & 14.5M & 3.6B & 799 \\
			Hollywood & Collab. & 1.1M & 56.3M & 4.9B & 2209 \\
			%Webuk & Web & 39.3M & 796.4M & & \\ 
			%Friendster & Social & 65M & 1.8B & & \\
			\bottomrule
			\end{tabularx}
			\begin{tablenotes}
				\item Datasets with the number of vertices, edges, triangles and the maximum trussness ($k_{max}$) in the largest weakly connected components without self edges. Sorted by the number of triangles.
			\end{tablenotes}
		\end{threeparttable}
\end{table}

We use 8 real-world graphs of different types as shown in the Table~\ref{table:datasets}. To simplify our experiments, we treat them as undirected, un-weighted graphs and only use the largest weakly connected component of each graph. We also removed all the self edges in each graph. 
%We sort the graph according to their number of triangles as ktruss community highly relies on triangles in the graph. 
All datasets are publicly available from Stanford Network Analysis Project\footnote{snap.stanford.edu} and Network Repository\footnote{networkrepository.com}.

\subsection{Index construction}
\label{eval_const}

\begin{table}
		\caption{Comparison of Index Construction}
		\label{table:index_construction}
		\centering
		%\begin{tabular}{|c|c|ccc|ccc|} \hline 
		%& Graph & \multicolumn{2}{|c|}{Index Size} & \multicolumn{2}{|c}{Index Time} \\
			%\cline{3-6}
			\begin{tabularx}{\linewidth}{c c *{6}{Y}}
			\toprule
			Graph & Decomp.
						& \multicolumn{3}{c}{Index Time (Sec.)} 
						& \multicolumn{3}{c}{Index Size (MB)} \\
			\cmidrule(lr){3-5} \cmidrule(l){6-8}
			 Name & Time (Sec.) & TCP & Equi & Our & TCP & Equi & Our \\ 
			\midrule
			Wiki & 239 & 139 & 63 & 83 & 58 & 25 & 32 \\ 
			Baidu & 742 & 494 & 269 & 350 & 306 & 179 & 237 \\
			Skitter & 366 & 167 & 151 & 139 & 139 & 240 & 193 \\ 
			Sinaweibo & 10728 & 11048 & 5724 & 6871 & 2744 & 1390 & 1810 \\
			Livejournal & 2201 & 1313 & 795 & 1020 & 1129 & 585 & 844 \\ 
			Orkut & 9028 & 7659 & 3609 & 5059 & 3302 & 1722 & 2479 \\
			Bio & 11239 & 13964 & 6223 & 8874 & 393 & 177 & 289 \\
			Hollywood & 14002 & 16620 & 4154 & 10182 & 1929 & 813 & 1276 \\ 
			
			%Wiki & 57.5 & 138.6 & 62.8 & 83.3 & 58.4 & 24.9 & 32.4 \\ 
			%Baidu & 224.5 & 493.9 & 268.5 & 350.3 & 305.8 & 179.0 & 237.3 \\
			%Skitter & 149.1 & 166.9 & 151.1 & 138.5 & 138.6 & 239.7 & 192.7 \\ 
			%Sinaweibo & 4049.9 & 11047.6 & 5724.3 & 6870.7 & 2743.8 & 1390.0 & 1810.4 \\
			%Livejournal & 627.6 & 1312.9 & 794.9 & 1020.0 & 1128.8 & 585.3 & 844.1 \\ 
			%Orkut & 1769.8 & 7659.4 & 3609.1 & 5058.7 & 3301.6 & 1721.8 & 2479.0 \\
			%Bio & 165.7 & 13963.9 & 6222.7 & 8873.6 & 393.1 & 176.6 & 289.4 \\
			%Hollywood & 791.7 & 16619.7 & 4154.4 & 10181.7 & 1928.7 & 812.5 & 1276.3 \\ 
			%Webuk & 13999 &   &  &  & \\ 
			%Friendster & 32364 &   &  &  & \\
			\bottomrule
		\end{tabularx}
\end{table}

We show in this section the index size and index construction time of the \twolevelindex{} compared to the TCP index and the Equitruss index in 
Table \ref{table:index_construction}. 
%Both indices are generated in memory and we show the size of the data structures that hold the index. 
We exclude the truss decomposition time for all three methods so that the index construction time only shows how long it takes to generate a certain index with edge trussness provided. 
We can see in Table \ref{table:index_construction} that the \twolevelindex{} has comparable construction time to the Equitruss index and both are faster than the TCP index. The index size of the \twolevelindex{} is smaller than the TCP index as there are no repeating edges stored in the index. However, the Equitruss has the smallest index size since it only stores edge list of the original graph while the \twolevelindex{} also stores the edges alongside vertices, which preserves the triangle connectivity inside k-truss communities. Note that if only \toplevelprob{} k-truss community queries are processed, the algorithm only needs to retrieve the top level index which has a much smaller size. %Note that the provided size is the minimum size to store the required index, the actual size may vary due to different implementations. 

\subsection{Query performance}
\label{eval_query_time}

In this section, we evaluate the query time of various query types to show the effectiveness of the \twolevelindex{}. As k-truss community query time heavily relies on the degree of query vertices, we use a similar procedure as used by \cite{huang2014querying} to partition vertices to be used in the experiments. Because only vertices with a degree of $k + 1$ can appear in a k-truss community with trussness of $k$. If we partition vertices uniformly by their degree and the graph has highly screwed degree distribution, then vertices in most of the partitions would have a too low degree to appear in a k-truss community. To show the performance of different algorithms on mining community structures in this kind of graphs, we fix the trussness of k-truss community search queries at $10$ and discard vertices with degree less than $20$. Then we uniformly partition the rest of vertices according to their degrees into 10 categories and at each category, we randomly select 100 sets of query vertices.

%\vskip 0.1in \noindent \textbf{Single vertex k-truss community search.}
\subsubsection{Single vertex k-truss community search.}
\label{eval_singlev_k_compare}

\begin{figure*}[t]
    \centering
    \includegraphics[width=0.8\textwidth]{./figures/singlev_k_compare.pdf}
    \caption{Comparison of single vertex k-truss community search of the \twolevelindex{}, the TCP index and the Equitruss index.}
    \label{fig:singlev_k_compare}
\end{figure*}

We first evaluate the single vertex k-truss community search performance and compare the query time with the TCP index and the Equitruss index. The results are shown in \autoref{fig:singlev_k_compare}. The \twolevelindex{} achieves best average query time for all graphs. It has an order of magnitude speedup compared to the TCP index for all graphs and $5\%$ to $400\%$ speedup compared to the Equitruss index for all graphs. However, the speed up is linear as all three indices have the same time complexity to handle single vertex k-truss community search queries. Note that very low average query time (around $10^{-5}$ second) means there is no vertex belonging to any k-truss community in that degree rank.

\begin{figure}[h]
\centering
\subfigure[Number of vertices in super-graphs.\label{fig:graphsize_vertices}]{\includegraphics[width=0.45\linewidth]{./figures/super_node_compare.pdf}}
\subfigure[Number of edges in super-graphs.\label{fig:graphsize_edges}]{\includegraphics[width=0.45\linewidth]{./figures/super_edge_compare.pdf}}
\caption{Super-graph size comparison of the \twolevelindex{} and the Equitruss index.}
\label{fig:graphsize}
\end{figure}

\vskip 0.1in \noindent \textbf{Reason of performance difference.} The main reason that the \twolevelindex{} is faster than the TCP index is the avoidance of the expensive BFS search during query time. The reason for performance differences of the \twolevelindex{} and the Equitruss index on various graphs is less obvious given that they both search for target communities on a super-graph of the original graph and then collect edges belonging to the target community. The difference lies in the fact that vertices and edges in the super-graph of the two indices represent different subgraphs and their relations of the original graph. Each vertex in the \twolevelindex{} represents a single k-truss community while vertices in the Equitruss index only represents a fraction of a k-truss community. So one vertice in \twolevelindex{} may be split into several vertices in the Equitruss index. 

We show the super-graph sizes of the \twolevelindex{} and the Equitruss index in \autoref{fig:graphsize}. We can see that the size of the super-graph of the Equitruss index is an order of magnitude larger than the super-graph of the \twolevelindex{}. The Equitruss index is slow while finding target communities due to the larger super-graph size. However, edge lists of a k-truss community can be more effectively retrieved as it is already stored in each super vertex. For the \twolevelindex{}, target communities are easier to identify, however, one need to iterate through the adjacent lists stored in super vertices to retrieve edges in the community. % and then convert vertices of the \inducedgraph{} back to their corresponding edges in the original graph. %Now it's not hard to understand the performance different on different real-world graphs, 

\subsubsection{The \toplevelprob{} query $vs.$ the \bottomlevelprob{} query (community search).}
\label{eval_top_bottom_compare}

\begin{figure*}[t]
    \centering
    \includegraphics[width=0.8\linewidth]{./figures/singlev_info_query.pdf}
    \caption{Three types (k-truss, max-k-truss, any-k-truss) of single-vertex \toplevelprob{} k-truss community query $vs.$ community search.}
    \label{fig:singlev_info_query}
\end{figure*}

%\begin{figure}[ht]
    %\centering
    %\includegraphics[width=\linewidth]{./figures/multiplev_2_info_query.pdf}
    %\caption{Single vertex query for exact truss community search.}
    %\label{fig:multiplev_2_info_query}
%\end{figure}

\begin{figure*}[t]
    \centering
    \includegraphics[width=0.8\linewidth]{./figures/multiplev_3_info_query.pdf}
    \caption{Three types (k-truss, max-k-truss, any-k-truss) multiple-vertex ($3$) \toplevelprob{} k-truss community query $vs.$ community search.}
    \label{fig:multiplev_3_info_query}
\end{figure*}

We perform all three basic types, \ie k-truss, max-k-truss and any-k-truss, of \toplevelprob{} k-truss community queries and perform community search queries on the targeting communities found by \toplevelprob{} queries. We show both single-query-vertex cases and multiple-query-vertex ($3$ vertices) cases in \autoref{fig:singlev_info_query} and \autoref{fig:multiplev_3_info_query}, respectively. 

%\vskip 0.1in \noindent \textbf{\toplevelprob{} query $vs.$ \bottomlevelprob{} query (community search).} 
We can see in both figures that our index is very effective for both \toplevelprob{} queries and community search queries. The average time for \toplevelprob{} queries spans from $1.22 x 10^{-5}$ second to $0.62$ second. The average time for community search queries is typically much higher than \toplevelprob{} queries since it needs to access edge level information, ranging from $2.20 x 10^{-5}$ to $979.81$ seconds depending on the size of target communities. The multi-hundred average run time comes from searching all truss communities that contain a query vertex (any-k-truss query) with a very high degree in the densest graph (bio). The fast query time of \toplevelprob{} queries makes it an excellent candidate for applications that require community-relation information such as whether a set of vertices belong to the same k-truss communities without digging into the details of any k-truss community. 

\subsubsection{K-truss query $vs.$ max-k-truss query $vs.$ any-k-truss query.}
\label{eval_k_type_compare}

%\vskip 0.1in \noindent \textbf{K-truss query $vs.$ max-k-truss query $vs.$ any-k-truss query.} 
We can also see in \autoref{fig:singlev_info_query} and \autoref{fig:multiplev_3_info_query} any-k-truss community search queries always have the highest average run time because it searches all the possible truss communities to which the query vertex/vertices belong. We can also see that k-truss community search queries usually have much smaller average run time than max-k-truss community search queries. 
%Is it because the index are more effective for k-truss community search query? Not really. 
%When checking the search result data, we find 
It is because that many k-truss queries fail to find a truss community as the query vertex/vertices do not belong to any truss community with the specified $k$, which is $10$ in our experiments. However, this problem is less severe for max-k-truss queries. Max-k-truss queries can always find a target community as long as the query vertex/vertices belong to any truss community. In most cases, max-k-truss queries can provide more useful information for applications that do not have much knowledge of the community structure in the underlying graph.

Another interesting trend is that as the degree of query vertex increases, the average run time for k-truss and any-k-truss community search queries increases, the average run time for max-k-truss queries decreases. The trend is caused by different reasons for three types of query. For k-truss queries, the average query time increases as the query vertex degree increases because that it is more likely to find a target k-truss community with the specified $k$, which is $10$ in our experiments. For max-k-truss queries, the average query time decreases as the query vertex degree increases because that target truss communities have higher trussness and smaller size. For any-k-truss queries, because the k-truss communities have hierarchical structures, more truss communities will be discovered by a query so that the average query time increases when the query vertex degree increases.

\subsection{\BottomLevelProb{} Query Analysis}
\label{eval_bottom_analysis}

%\begin{figure}[h]
%\centering
%\subfigure[Boundary search.\label{fig:boundary}]{\includegraphics[width=0.78\linewidth]{./figures/boundary_size.pdf}}
%\subfigure[Maximim triangle connected path.\label{fig:path}]{\includegraphics[width=0.18\linewidth]{./figures/path.pdf}}
%\caption{Queries inside communities.}
%\label{fig:inside_query}
%\end{figure}

\subsubsection{K-truss community boundary search.}

\begin{figure}[ht]
    \centering
    \includegraphics[width=\linewidth]{./figures/boundary_size.pdf}
    \caption{Randomly sampled boundary length for k-truss communities with different trussness.}
    \label{fig:boundary}
\end{figure}

%\vskip 0.1in \noindent \textbf{Boundary search.} 
We randomly select $1000$ query vertices from various degree buckets and perform the boundary search for the k-truss community with highest trussness that contains each query vertex and the trussness of the community and their boundary length in \autoref{fig:boundary}. We can see that in many graphs there is a huge k-truss community of size several magnitude larger than other smaller k-truss communities. This community usually have a hierarchical structure, \ie larger k-truss communities with low trussness contain smaller k-truss communities with high trussness. \autoref{fig:boundary} also shows that the upper bound of the boundary length of k-truss communities decreases as the trussness increases. The main reason for this is that sizes of high trussness k-truss communities are usually smaller than sizes of low trussness k-truss communities. However, the lower bound of the boundary length of k-truss communities increases as the trussness increases. The reason is that there are many small-size k-truss communities which are triangle connected to very few other k-truss communities, \ie they are like isolated islands of the graph and many of them haven't formed a hierarchical structure. 

\subsubsection{Triangle connected maximin path search.}

\begin{figure}[ht]
    \centering
    \includegraphics[width=0.5\linewidth]{./figures/path.pdf}
    \caption{Average query time for triangle connected maximim path search.}
    \label{fig:path}
\end{figure}

We randomly select $1000$ pair of vertices from various degree buckets and show the average query time for the triangle connected maximin path search with them in \autoref{fig:path}. 
%\vskip 0.1in \noindent \textbf{Maximin path search.} 
The figure clearly shows that as the degree of vertex increases, the query time decreases. The reason is that for a pair of vertices with high degree, it is more likely that they belong to the same k-truss community with higher trussness and smaller size. So there is no surprise that the query vertices are closer to each other and a triangle connected maximim path between them tends to be shorter. We notice that the triangle connected maximin path search have much higher average query times for the same graph than k-truss community search because it needs to run a BFS traversal inside a target community which is very time-consuming.

\section{Related Works}
\label{relatedwork} 

Our work falls in the category of cohesive subgraph mining \cite{sozio2010community,cui2014local,li2015influential,bera2018maximal} such as community detection and community search. It's most related to the inspiring work \cite{huang2014querying}, which introduce the model of k-truss community based on triangle connectivity. Triangle connected k-truss communities mitigate the "free-rider" issue but at the cost of slow computation efficiency especially for vertices belongs to large k-truss communities. To speed up the community search based on this model, an index structure called the TCP index is proposed in \cite{huang2014querying} that each vertex holds their maximum spanning forest based on edge trussness of their ego-network. 
%Triangle connected k-truss communities mitigate the "free-rider" issue but at the cost of slow computation efficiency especially for vertices belongs to large k-truss communities so that they are not able to meet requirements of queries in some cases. 
Later, the Equitruss index \cite{akbas2017truss} is proposed that use a super-graph based on truss-equivalence as an index to speed up the single vertex k-truss community search.  
%stores a partition of the original graph based on the notion of k-truss equivalence. 
The Equitruss index is similar to our index structure in the sense that it also use a super-graph for the index. However, the vertex in the super-graph of the Equitruss index is a subgraph of a k-truss community while an edge represents triangle connectivity. Our \twolevelindex{} contains a more compact super-graph that a vertex contains a k-truss community and edges represent k-truss community containment relations. We have used both TCP index and Equitruss index as comparisons in our evaluation.

 
%Many previous works have studied this problem based on various cohesive subgraph models, such as clique (\cite{bron1973algorithm,rossi2014fast}), k-core (\cite{shin2016corescope,barbieri2015efficient,li2017discovering}), k-truss (\cite{huang2014querying,wang2012truss,cohen2008trusses,huang2015approximate,huang2016truss,zheng2017finding}), k-plex (\cite{wangquery}) and quasi-clique (\cite{tsourakakis2013denser, lee2016query}). The k-truss concept is first introduced by the work \cite{cohen2008trusses}. However, the original definition of a k-truss lacks the connectivity constraint so that a k-truss may be a unconnected subgraph. \cite{huang2014querying} introduce the model of k-truss community based on triangle connectivity. The notion of triangle connected k-truss communities is also referred to as $k-(2,3)$ nucleus in \cite{sariyuce2016fast, sariyuce2017nucleus} where they propose an approach based on the disjoint-set forest to speed up the process of nucleus decomposition. K-truss decomposition is also studied in \cite{huang2016truss, zou2017truss} for probabilistic graphs. 

%An $\alpha$-adjacency $\gamma$-quasi-$k$-clique model is introduced by \cite{cui2014local} for online searching of overlapping communities. $\rho$-dense core is a pseudo clique recently introduced by \cite{koujaku2016dense} that can deliver the optimal solution for graph partition problems. The pattern of k-core structures is studied by \cite{shin2016corescope} for applications such as finding anomalies in real world graphs, approximate degeneracy of large-scale graphs and so on. \cite{li2015influential} introduce a novel community model called k-influential community based on the concept of k-core, which can capture the influence of a community. %\cite{chen2016efficient,chang2017mathsf,li2017finding,bi2017optimal,li2017most} also study this.
%\cite{wu2015robust} systematically study the existing goodness metrics and provide theoretical explanations on why they may cause the free rider effect. %We further develop a query biased node weighting scheme to reduce the free rider effect.
%\cite{huang2015approximate} try to address the "free rider" issue by finding communities that meet cohesive criteria with the minimum diameter.
%There are also many works studied community related problems on attribute graph (\cite{fang2016effective,shang2016agar,shang2017attribute,huang2017attribute}), \eg finding communities that satisfy both structure cohesiveness, \ie its vertices are tightly connected, and keyword cohesiveness, \ie its vertices share common keywords.
%\cite{interdonato2017local} design a framework for local community detection in multilayer networks.
\section{Conclusion}
\label{conclusion}

In this work, we use information required to process a community query to divide local k-truss community queries into two categories, the \toplevelprob{} query and the \bottomlevelprob{} query. We designed \twolevelindex{} that stores the \treeindex{} in the top level index for locating relevant communities and the \inducedgraph{} in the bottom level index to preserve the triangle connectivity at the edge level inside each k-truss community. 
We proved the effectiveness of our index structure theoretically and experimentally for processing both \toplevelprob{} queries and \bottomlevelprob{} queries with a single query vertex or multiple query vertices. We compared with state-of-the-art methods for single-vertex k-truss community search and showed that our method has the best performance. %Results show our method can process \toplevelprob{} queries in the range of hundreds of micro seconds to less than a second and \bottomlevelprob{} queries from a few seconds to hundreds of seconds for highest degree vertices within large communities.

%\begin{acknowledgements}
%If you'd like to thank anyone, place your comments here
%and remove the percent signs.
%\end{acknowledgements}

% BibTeX users please use one of
\bibliographystyle{spbasic}      % basic style, author-year citations
%\bibliographystyle{spmpsci}      % mathematics and physical sciences
%\bibliographystyle{spphys}       % APS-like style for physics
\bibliography{reference}   % name your BibTeX data base

\end{document}
% end of file template.tex

