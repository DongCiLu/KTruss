\section{Query on \TwoLevelIndex{}}
\label{query}

We classify k-truss community related queries into two categories according to the level of information required. The \toplevelprob{} query (Section~\ref{\toplevelprob{}}) requires only information of relations between k-truss communities and to which k-truss communities query vertices belong. For example, "Do query vertices belongs to same k-truss communities?", "To which k-truss communities query vertices belong have the highest trussness?". This type of queries can be answered solely by the top level index of our \twolevelindex{}. Another type of queries, which requires edge level information to process, is called the \bottomlevelprob{} query (Section~\ref{\bottomlevelprob{}}). For example, the widely studied community search queries. We process this type of queries by first locating the target k-truss communities with the top level index and then diving into the edge-level details with the bottom level index. 

\subsection{The \TopLevelProb{} Query}
\label{\toplevelprob{}}
%K-truss community info queries have the coarsest granularity, \ie it only ask for information on the community level and don't require any details inside a k-truss community.  Queries like these can be directly answered by only looking up in the top level \treeindex{} of the \twolevelindex{}. Since each vertex contains meta-data that stores properties of the corresponding k-truss community, this type of query can also answer queries about the trussness of communities, the size of communities, etc.

The \twolevelindex supports three basic types of \toplevelprob{} k-truss community queries of a single or a set of query vertices listed below: 
\begin{itemize}
	\item{K-truss query:} Given a set of vertices $Q$ and an integer $k$, find all the k-truss communities that contain $Q$ with trussness of $k$.
	\item{Max-k-truss query:} Given a set of vertices $Q$, find k-truss communities that contain $Q$ with the highest possible trussness.
	\item{Any-k-truss query:} Given a set of vertices $Q$, find all the k-truss communities that contain $Q$.
\end{itemize}

\begin{algorithm}
	\KwData{$G^{o}(V^{o},E^{o})$, $G^{c}(V^{c},E^{c})$, $H$, $Q$}
	\KwResult{subgraph $S$ of $G^c$}
	\SetKwProg{Fn}{function}{}{end}
	\BlankLine
	$S \gets \emptyset$\;
	$initialized \gets false$\;
	\For{$u^o \in Q$}{
		$SS \gets$ singlev\_subgraph($u^o$)\;
		\lIf{$!initialized$} {
			$S \gets SS$, $initialized \gets true$
		}
		\lElse {
			$S \gets S \cap SS$
		}
	}
	\Return{$S$}
	\BlankLine
	\Fn{branch\_search ($u^s$)}{
		$B \gets \emptyset$\;
		\While{$u^c \neq null$} {
			$B \gets B \bigcup u^s$\;
			$u^c \gets u^{c}.parent$\;
		}
		\Return{$B$}
	}
	\BlankLine
	\Fn{singlev\_subgraph ($u^o$)}{
		$SS \gets \emptyset$\;
		\For{$v^o \in N_u$} {
			$u^s \gets H[(u^o, v^o)]$\;
			$B \gets$ branch\_search($u^c$)\;
			$SS \gets SS \bigcup B$\;
		}
		\Return{$SS$}
	}
	\caption{$union-intersection$ Algorithm.}\label{alg:union_intersection}
\end{algorithm}

All three types of \toplevelprob{} k-truss community queries share a similar querying process on the top level index of the \twolevelindex{}, call $union-intersection$ procedure. It first takes the $union$ subgraphs of the \treeindex{} that their corresponding k-truss communities in the original graph contain edges of a single query vertex; then it calculates the $intersection$ of subgraphs of each query vertex. Finally, from the $intersection$ of subgraphs of the \treeindex{}, we can retrieve vertices that have weights of a specified $k$ (k-truss query), calculate vertices that have maximum weights (Max-k-truss query) or return all the vertices (Any-k-truss query). The corresponding k-truss communities in the original graph are used to answer the query. \autoref{alg:union_intersection} shows the detailed steps of the $union-intersection$ procedure.

%\begin{Thm}
%The union of all communities $\bigcup C_i$ found by \autoref{alg:\inducedgraph{}_query} is the union of all the k-truss communities containing query vertex $q$.
%\label{thm:\inducedgraph{}_query}
%\end{Thm}
%
%\begin{proof}
%According to \autoref{def:\inducedgraph{}}, a vertex $x$ in \inducedgraph{} $G_m$ with weight $w_{x} \le k$ means the represented edge $e$ in $G_o$ has trussness $\tau_{e} \le k$ and thus can be included in a k-truss community. An edge $(x,y)$ in \inducedgraph{} $G_m$ with weight $w_{(x,y)} \le k$ means the represented triangle $\triangle$ in $G_o$ has all three edges with trussness higher or equal to $k$ and thus the triangle is included in a k-truss community containing all three edges of it. Adjacent edges in $G_m$ means adjacent triangles in $G_o$ and connected components in $G_m$ means triangle connected components in $G_o$. So, BFS/DFS search starts with a seed vertex $x$ with weight constraint will find the maximal connected component including $x$ which representing the k-truss community that $e$ belongs to in $G_o$ ($x$ represents $e$ in $G_m$). Therefore, performing such BFS/DFS searches on each edge of the query vertex will find all the k-truss communities that the query vertex belongs to.
%\end{proof}

The $branch\_search$ in the \autoref{alg:union_intersection} is to find all the truss communities that to which an edge in the original graph belongs. The time and space complexity for this step is$\tau_e$. The $singlev\_subgraph$ iterates all the adjacent edges of a query vertex to discover the subgraph in $G^c$ that contains it. The time and space complexity is $\sum_{v \in N_u}{\tau_{(u,v)}}$. Finally, the algorithm needs to find subgraph for each query vertex, so the total time and space complexity is $\sum_{u \in Q}{\sum_{v \in N_u}{\tau_{(u,v)}}}$. 
%It is the time and space complexity for all three types of k-truss community info queries and they all can be answered by checking the results one pass.

\subsection{The \BottomLevelProb{} Query}
\label{\bottomlevelprob{}}

The \bottomlevelprob{} k-truss community query requires information of finest granularity as it needs to explore the inner structure of a k-truss community. Our bottom level index contains the detailed triangle connectivity information that makes such queries possible. To process a k-truss community query, we first locate the target k-truss communities with the top level index and then compute query results using edge-level details provided by the bottom level index.  We show three concrete examples of this type of queries in this section: the k-truss community search (Section~\ref{k-truss community search}), the k-truss community boundary search (Section~\ref{boundary search}) and the triangle connected maximin path search (Section~\ref{path search}).

\subsubsection{K-truss community search}
\label{k-truss community search}

The k-truss community search is the simplest form of the \bottomlevelprob{} k-truss community query. First, the $union-intersection$ algorithm is performed to get target communities of the query. Then for each community in target communities, we collect edges contained in it by gathering the vertex list of subgraphs of $G^m$ stored alongside $G^c$ vertices. Finally, edges of the original graph $G^o$ can be retrieved by converting their corresponding vertices in $G^m$. \autoref{alg:search_query} shows the details.
%As we mentioned in section \ref{structure}, the partial $G^m$ is stored as adjacent list. We only need to collect all the vertices of the partial $G^m$, which represent edges in the orignal graph $G^o$. 

\begin{algorithm}
	\KwData{$G^{o}(V^{o},E^{o})$, $G^{m}(V^{m},E^{m})$, $G^{c}(V^{c},E^{c})$, $H$, $Q$}
	\KwResult{all k-truss communities $C$ containing $Q$}
	\BlankLine
	$S \gets union-intersect$($G^o$, $G^c$, $H$, $Q$)\;
	\For{$v^c \in S$} {
		$C_i \gets \emptyset$\;
		\For{$v^m \in v^{c}.adj\_list.keys$}{
			find corresponding $e^o$ of $v^m$;
			$C_i \gets C_i \bigcup e^o$\;
		}
	}
	\Return{$\bigcup{C_i}$}
	\caption{K-truss Community Search Query}\label{alg:search_query}
\end{algorithm}

The $union-intersect$ algorithm takes $\sum_{u \in Q}{\sum_{v \in N_u}{\tau_{(u,v)}}}$ time and space. Each edge in the target communities will only be accessed once, so the time and space complexity for the search are $\sum_{u \in Q}{\sum_{v \in N_u}{\tau_{(u,v)}}} + |\bigcup{C_i}|$.

%The intra-community query requires information of finest granularity as it needs to explore the inner structure of a k-truss community. Our bottom level index, \inducedgraph{}, contains the detailed triangle connectivity information that make such queries possible. For some queries, the top level \treeindex{} may also help. We show two concrete examples of such queries in this section: finding boundaries among k-truss communities and finding a triangle connected maximin trussness path between two query vertices, etc. % For the sake of space, we omitted the detailed algorithm here.

\subsubsection{k-truss community boundary search}
\label{boundary search}

The boundary of a k-truss community $C_i$ contains edges in the community that have triangle adjacent neighbor edges belong to other k-truss communities $C_j, C_j \notin C_i$. To find the boundary of a k-truss community, as vertices in the \treeindex{} $G^c$ have subgraphs of the \inducedgraph{} $G^m$ stored in it, we can first gather the subgraph $S$ of $G^o$ that contains children vertices of the corresponding vertex $v^c$ of the queried k-truss community $C$ using the top level index. Then we iterate through all the vertices of the subgraph of $G^m$ that stored in $S$ and select vertices that have neighbors stored in other $G^c$ vertices $u^c$ that $u^c \notin S$. Finally, the collected vertices of $G^m$ can be mapped back to edges of the original graph $G^o$. Since the search exams all the edges in the queried community, the algorithm's time complexity is $O(|C|)$, and the space complexity is $O(|B|)$ , where $B$ denotes the returned boundary. For the sake of space, we omitted the detailed algorithm here.

\subsubsection{Triangle connected maximin path search}
\label{path search}

Given two query vertices, a triangle connected maximin path is a path connecting two queries vertices that has all the edges are triangle connected and maximizes the minimum edge trussness. To find such a path, the algorithm first perform a max-k-truss \toplevelprob{} query to find communities containing both query vertices that have the highest trussness, denoted by $C_{max\tau}$. Then in one of the target communities, the algorithm starts a breadth first search that amid to find a path connecting any pair of edges of the source and target vertices. Due to the property of a maximum spanning tree, the found path is a maximin path of edge trussness. Such a path is guaranteed to exist as long as a max-k-truss community containing both the source and target vertices can be found because edges belonged to the same k-truss community is triangle connected. The algorithm performs a BFS after a max-k-truss info query, so the time complexity is $\sum_{v \in N_{src}}{\tau_{(src,v)}} + \sum_{v \in N_{dst}}{\tau_{(dst,v)}} + \min_{C \in C_{max\tau}}{|C|}$ and space complexity is $O(|P|)$ , where $P$ denotes the returned path. For the sake of space, we omitted the detailed algorithm here.
%starts a breadth first search at a vertex in the stored partial $G^m$ that has the highest degree. The search ends when it reach both a vertex in $G^m$ representing an adjacent edge $e_{src}$ of the source vertex in $G^o$ and a vertex in $G^m$ representing an adjacent edge $e_{dst}$ of the target vertex $v_{dst}$. The algorithms performs a BFS after a max-k-truss info query, so the time complexity is $\sum_{v \in N_{src}}{\tau_{(src,v)}} + \sum_{v \in N_{dst}}{\tau_{(dst,v)}} + \min_{C \in C_{max\tau}}{|C|}$.
