\section{Related Works}
\label{relatedwork} 

Our work is most related to the inspiring work~\cite{huang2014querying} which introduce the notion of k-truss community based on triangle connectivity. An index structure call TCP is proposed in~\cite{huang2014querying} that each vertex holds their maximum spanning forest based on edge trussness of their ego-network. Triangle connected k-truss communities mitigate the "free-rider" issue but at the cost of slow computation efficiency especially for vertices belongs to large k-truss communities so that they are not able to meet requirements of queries in some cases. We have use this work as a comparison in section \autoref{evaluation}.

The notion of triangle connected k-truss community is also referred to as $k-(2,3)$ nucleus in~\cite{sariyuce2016fast} where they propose an approach based on disjoint-set forest to speed up the process of nuclues decomposition. We are more emphasize on the fast query process rather speed up the truss decomposition. Also the \treeindex{} is easier to maintain for dynamic graphs.

Our work falls in the category of cohesive subgraph mining~\cite{huang2017community,koujaku2016dense,sozio2010community,cui2014local,li2015influential,cui2013online,mcauley2012learning}, such as clique~\cite{bron1973algorithm,rossi2014fast}, k-core~\cite{cheng2011efficient,shin2016corescope,barbieri2015efficient,li2017discovering}, k-truss~\cite{huang2014querying,wang2012truss,cohen2008trusses,huang2015approximate,huang2016truss,zheng2017finding}, k-plex~\cite{wangquery} and quasi-clique~\cite{tsourakakis2013denser, lee2016query}. An $\alpha$-adjacency $\gamma$-quasi-$k$-clique model is introduced by~\cite{cui2014local} for online searching of overlapping communities. $\rho$-dense core is a pseudo clique recently introduced by~\cite{koujaku2016dense} that is able to deliver optimal solution for graph partition problems. The pattern of k-core structures is studied by~\cite{shin2016corescope} for applications such as finding anomalies in real world graphs, approximate degeneracy of large-scale graphs and so on. K-truss decomposition is also studied in~\cite{huang2016truss, zou2017truss} for probabilistic graphs.

\cite{li2015influential} introduce a novel community model called k-influential community based on the concept of k-core, which can capture the influence of a community. \cite{chen2016efficient,chang2017mathsf,li2017finding,bi2017optimal,li2017most} also study this.

\cite{wu2015robust} systematically study the existing goodness metrics and provide theoretical explanations on why they may cause the free rider effect. We further develop a query biased node weighting scheme to reduce the free rider effect. 

\cite{huang2015approximate} tries to address the "free rider" issue by finding communities that meet cohesive criteria with the minimum diameter.

\cite{fang2016effective} works on attribute graph, finding communities that satisfies both structure cohesiveness (i.e., its vertices are tightly connected) and keyword cohesiveness (i.e., its vertices share common keywords). \cite{huang2016attribute, shang2016agar, shang2017attribute,zhang2017visualizing,fang2017effective,huang2017attribute} also works on this.

\cite{sariyuce2017nucleus} nucleus decompositions