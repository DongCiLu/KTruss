\section{Evaluations}
\label{evaluation}

In this section, we show the results of experimental evaluation of truss community query on our index structure.

\subsection{Datasets}
\label{eval_datasets}

\begin{table}
		\caption{Datasets}
		\vspace{2 mm}
		\label{table:datasets}
		\begin{threeparttable}
			\centering
			\begin{tabular}{c|cccc} \hline
				Dataset & Type & $|V_{wcc}|$ & $|E_{wcc}|$ & $\overline{\sigma}$ \\ \hline
				Wiki & Communication & 2.4M & 4.7M & 3.9 \\ 
				Skitter & Internet & 1.7M & 11.1M & 5.07 \\ 
				Livejournal & Social & 4.8M & 43.4M & 5.6 \\ 
				Hollywood & Collaboration & 1.1M & 56.3M & 3.83 \\ 
				Orkut & Social & 3M & 117M & 4.21 \\ 
				Sinaweibo & Social & 58.7M & 261.3M & 4.15 \\ 
				Webuk & Web & 39.3M & 796.4M & 7.45 \\ 
				Friendster & Social & 65M & 1.8B & 5.03 \\ \hline
			\end{tabular}
			\begin{tablenotes}
				\item Datasets with the number of vertices and edges in the largest weakly connected components, and the average shortest distance $\overline{\sigma}$ of 100,000 vertex pairs.
			\end{tablenotes}
		\end{threeparttable}
\end{table}

We evaluate our algorithm on 8 graphs from different disciplines as shown in table~\ref{table:datasets}. All graphs are complex networks that have power-law degree distributions and relatively small diameters. To simplify our experiments, we treat them as undirected, un-weighted graphs and only use the largest weakly connected component of each graph. All datasets are collected from Snap~\cite{snapnets}.

\subsection{Experiment settings}
\label{eval_system}

We evaluate our algorithms, we use a Cloudlab~\cite{RicciEide:login14} c8220 server with two 10-core 2.2GHz E5-2660 processors and 256GB memory. We base our algorithm on Snap~\cite{snapnets}. All algorithms are implemented in C++. 

\subsection{Query time}
\label{eval_query_time}

We first compare the query time of truss community search based on our index structure with the state-of-art related work.
 
\subsection{Index construction time and size}
\label{eval_const}

We show in this section that our index achieves the similar construction time and size as previous work.

\subsection{Index update time}
\label{eval_update}

The index update time upon graph changes, i.e. edge deletion, is shown in ... 


