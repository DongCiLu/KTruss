\section{Related Works}
\label{relatedwork} 

Our work is most related to the inspiring work~\cite{huang2014querying} which introduce the notion of k-truss community based on triangle connectivity. An index structure call TCP is proposed in~\cite{huang2014querying} that each vertex holds their maximum spanning forest based on edge trussness of their ego-network. Triangle connected k-truss communities mitigate the "free-rider" issue but at the cost of slow computation efficiency especially for vertices belongs to large k-truss communities so that they are not able to meet requirements of queries in some cases. We have use this work as a comparison in section \autoref{evaluation}.

The notion of triangle connected k-truss community is also referred to as $k-(2,3)$ nucleus in~\cite{sariyuce2016fast} where they propose an approach based on disjoint-set forest to speed up the process of nuclues decomposition. We are more emphasize on the fast query process rather speed up the truss decomposition. Also the \treeindex{} is easier to maintain for dynamic graphs.

Our work falls in the category of cohesive subgraph mining~\cite{koujaku2016dense, sozio2010community, cui2014local, li2015influential, cui2013online, mcauley2012learning}, such as clique~\cite{bron1973algorithm, rossi2014fast}, k-core~\cite{cheng2011efficient, shin2016corescope}, k-truss~\cite{huang2014querying, wang2012truss, cohen2008trusses, huang2015approximate, huang2016truss} and quasi-clique~\cite{tsourakakis2013denser}. An $\alpha$-adjacency $\gamma$-quasi-$k$-clique model is introduced by~\cite{cui2014local} for online searching of overlapping communities. $\rho$-dense core is a pseudo clique recently introduced by~\cite{koujaku2016dense} that is able to deliver optimal solution for graph partition problems. The pattern of k-core structures is studied by~\cite{shin2016corescope} for applications such as finding anomalies in real world graphs, approximate degeneracy of large-scale graphs and so on. K-truss decomposition is also studied in~\cite{huang2016truss} for probabilistic graphs.

